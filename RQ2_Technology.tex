\subsubsection{Software development tools and techniques used (D2)} \label{sec:rq2-d2}
\begin{table}[!ht]
\caption{Comparison of tech platforms, OS and programming languages between our findings and prior findings}
\begin{tabular}{llll}

\hline
\multicolumn{1}{c}{\textbf{Practices}} & \multicolumn{1}{c}{\textbf{Our Study}} & \multicolumn{1}{c}{\textbf{Prior Study}} & \multicolumn{1}{c}{\textbf{Comparison}}\\ 
\hline 


\multicolumn{1}{l|}{\multirow{3}{*}{\parbox{0.14\textwidth}{Which implementation technologies and tools are adopted by software development professionals?}}
} 

&

\multicolumn{1}{l|}{\multirow{3}{*}{\parbox{0.26\textwidth}{
\vspace{-45pt} (1) Majority of our participants were related to the various types of web-based applications that convey the market demand for web-based apps in Bangladesh. (2) Linux is mainly used for development purposes while macOS is highly preferred by managers. (3) Javascript and Java are two mostly used programming languages in the Bangladesh SE industry since the demand for web-based services as well as mobile apps.
}}} 
& 
\multicolumn{1}{l|}{\comparisoncell{0.25}{{\vspace{40pt} There exists a high level focus on web-based platforms in the software industry of New Zealand \citep{Wang2018}.
}}}
& 
\multirow{3}{*}{\parbox{0.21\textwidth}{
We have observed that web-based platforms have widespread demand among different countries along with Bangladesh. Besides, in using programming languages like Java and python, Bangladesh has similarities with Turkey but is different from New Zealand.
}} \\ \cline{3-3}

\multicolumn{1}{l|}{}                                       
& 
\multicolumn{1}{l|}{}                                       
& 

\multicolumn{1}{l|}{\comparisoncell{0.25}{{
\vspace{40pt} Windows is highly preferable among developers of New Zealand \citep{Wang2018}.
}}}                                                         
&                                                       
\\ \cline{3-3}

\multicolumn{1}{l|}{}               
& 
\multicolumn{1}{l|}{}                                        
& 

\multicolumn{1}{l|}{\comparisoncell{0.25}{
\vspace{40pt} The use of Java as a programming language is spacious in Turkey \citep{Garousi2015}. However, in New Zealand, the usage rate of Java, as well as python ranks somewhat low \citep{Wang2018}.
}} 
&                                                           
\\ \hline

\end{tabular}
\label{table:tech_comparison}
\end{table}

We compare our observations from three questions: \begin{inparaenum}
\item Technology Platform (Q9).
\item Operating System (Q10).
\item Programming Language (Q11).
%\item Framework (Q12).
%\item IDE (Q13).
\end{inparaenum} The observations from two questions (Programming language Q11 and IDE used Q12) could not be compared, because 
those were not previously asked in the context of other countries. We summarize the comparisons in Table \ref{table:tech_comparison} and discuss those below.

% To find out the overall comparison of this section, we report the following sub-sections:
% \begin{itemize}
% \item Technology Platform (Q9).
% \item Operating System (Q10).
% \item Programming Language (Q11).
% \end{itemize}
% 
% We have compared our study findings on these sub-topics to get similarities and dissimilarities with the countries like Turkey, New Zealand, etc., on which a couple of relevant studies have been conducted. We compare the major findings of our study with the similar findings of previous studies in Table \ref{table:tech_comparison}.
%\paragraph{Technology Platforms}

\nd\bf{$\bullet$ Technology platforms (Q9).} As shown in
Figure~\ref{fig:platforms}, most of our survey respondents (80\%) work in web
platforms. This outcome is similar to the result of the survey of Wang et al.
\citep{Wang2018} in which the authors found that most of their respondents also
develop in web platforms in New Zealand.


%\paragraph{Operating Systems}
\nd\bf{$\bullet$ Operating Systems (Q10).} We have found an interesting point
that Windows is mostly used among developers of New Zealand based on the study
\citep{Wang2018} nevertheless, Linux is mostly used in the case for Bangladeshi
developers corresponding to our survey presented in Figure~\ref{fig:os}.

%\paragraph{Programming Languages}
\nd\bf{$\bullet$ Programming Languages (Q11).} According to the study of Wang et
al.\citep{Wang2018}, Java ranks quite low in New Zealand; nevertheless, it is
the second most used programming language in Bangladesh as per our study
reported in Figure~\ref{fig:languages} as well as the most used language in
Turkey \citep{Garousi2015}. Again, Python did not have a good standing in the
ranking of languages used in New Zealand; nevertheless, it is used significantly
in Bangladesh.

\begin{tcolorbox}[flushleft upper,boxrule=1pt,arc=0pt,left=0pt,right=0pt,top=0pt,bottom=0pt,colback=white,after=\ignorespacesafterend\par\noindent]
\nd\it{\bf{RQ2-D2. Software development tools and techniques used.}}
Like other SE industries, the web is the main technology platform in Bangladesh. Linux is the preferred OS in Bangladesh's SE industry, while it is  Windows in  New Zealand SE industry. Although Java and Python are popular languages in the SE industry in Bangladesh, we have noticed that they are not very popular in the New Zealand SE industry.\gias{what is slightly different?}\partha{Updated}
\end{tcolorbox}