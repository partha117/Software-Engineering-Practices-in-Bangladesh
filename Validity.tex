\section{Threats to Validity}
\label{validity}

\subsection{Construct Validity}
% Construct validity threats concern the relation between theory and
% observations. 
Construct validity is mainly concerned with the extent to which the
study objectives truly represent the theory behind the study\cite{Wohlin2012}. In our study, we have used open coding strategy to label the survey responses. The nature of this coding strategy may introduce researcher bias into coded labels. To mitigate the issue, the labels have been coded by two individuals, and the codes are accepted when there is a reasonable agreement among the coders. Another issue can be whether our data actually represents real-world SE practices. This study counted the votes and made statistical inference, which is common in survey-based studies. It is believed that voting data can, to a certain extent, reflect the opinions of the majority. It was previously observed\cite{Garousi2015} that people tend to form their answers close to expected answers when evaluated. To mitigate the threat, before the survey, we informed participants that our motive in this survey was to get a decent understanding of current practices, and we do not intend to collect any personally identifiable data. Construct threats may also be introduced by a misleading interpretation of the survey questions. We conducted a preliminary survey and interview session with some participants to rule out any ambiguity from survey questions and thus tried to reduce such risk.

\subsection{Internal Validity}
% Threats to internal validity refer to how well the research is
% conducted.
Internal validity is a property of scientific studies that refers to how well a study has been conducted. A threat to internal validity in this study is inherent in the participant selection bias. We used several social platforms, personal connections to reach as many participants as possible. Another threat could arise from the placement of the options in a multiple-choice question. It is often observed that survey participants often show bias towards the first option in any multiple-choice question\cite{Uddin2019}. However, in one of the multiple-choice questions (Q9), the `Web' option was placed at the bottom of the list. Despite this placement, we observed most of the participants selected `Web' as a technology platform. But practically, from the personal experience of the authors, there is no bias in this opinion.


\subsection{External Validity}
% Threats to external validity compromise the confidence in stating
% whether the study results apply to other groups.
External validity is concerned with the generalization of the study result. In our study, we have participants from almost all the groups of the Bangladesh software industry. However, it is difficult to claim the statistical generalizability of our findings, given that our sample included 137 respondents where there 1100+ companies and 3,00,000 IT professionals\cite{BASIS2018} in the software industry of Bangladesh. Moreover, emerging IT industries share a common trend of challenges\cite{Sison2006, lloyd2020}. Thus, our findings are also applicable to other emerging software industries across the globe.