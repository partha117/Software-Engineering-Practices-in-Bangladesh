\section{Introduction}
\label{introduction}
\newcolumntype{v}{>{\hsize=.08\hsize}X}
\newcolumntype{b}{>{\hsize=.77\hsize}X}
\newcolumntype{m}{>{\hsize=.15\hsize}X}
 \newcolumntype{y}{>{\hsize=.33\hsize}X}
%The software engineering (SE) industry in Bangladesh is considered relatively small compared to the size of its population (160 million-plus) and the size of the national economy. %Yet, the software industry in this country has started rapidly emerging and become part of Bangladesh's rapid development in recent years. 
%The majority of software companies were established after 2000 and have become a crucial resource for the economy of Bangladesh \citep{Basis2017}, and also the growth rate is expected to continue sharply. This propitious growth is bolstered by export trends and a large demand for IT automation in the domestic market. To continue this achievement by fulfilling the growing expectations of software consumers, the SE industry needs to follow standard practices in producing good quality software products. %Thus for this, we have to first point out how the software development process is exercised in such a rapidly emerging industry. 

%The practice of software development varies from region to region. A certain methodology may become more efficient in some regions but is not considered efficient in other regions. Although most of the factors behind software development are the same all the time, there are certain points that vary from region to region. For software development professionals, deciding on processes, practices, and techniques is critical. Sometimes, these decisions might be based on marketing and literature bias that supports the new or industry-supported practice.

%Many studies concerning the practices of software development have been conducted. Those studies convey similarities, but also differences in diverse regions, countries, and communities. Previous empirical studies have gone into how software development practices were conducted in North America, Europe, Turkey, New Zealand, etc. There are some prior studies on the Bangladesh software engineering (SE) industry, but the studies' focus and dimension were different. To the best of our knowledge, this is the first study to identify software engineering practices in emerging countries by taking the software industry in Bangladesh as a case study on how the software industry in an emerging environment works, the challenges they face, and the practices they adopt.
% The delivery of quality outcomes on time with a competitive budget is a major
%  challenge for the software industry. The practice of state-of-the-art SE
%  methods and technologies is indispensable to achieve success. 
 A study on the software development practice in a country
can reveal the challenges and scopes of improvement for software industries of similar countries. There are several studies on the software
industry of countries that focused on understanding the software development practices 
in developed countries and/or countries with matured software industry (e.g., Japan, Canada)~\citep{Garousi2013, Garousi2015, Vonken2012, Wang2018}. However, 
the software industry of a developed country differs from that of a developing
country in many aspects. Easy migration opportunities of the developers cause
constant scarcity of experienced people in developing countries. Still, due to
the abundance of Computer Science graduates, the software industry of some
developing countries are continuing to emerge by catering to the majority of the
local market as well as expanding in the global market.


A recent study 
focused on the software development ecosystems in Malaysia~\citep{Baharom2006}. However, the study 
did not consider the diverse development practices (e.g, security, scalability in development practices). There 
is also no study on a systematic comparison of the software development practices in the software industries 
between the developing and developed countries. In this paper, we focus on understanding the software development 
practices in an emerging software industry 
by analyzing those practices in another developing country, Bangladesh. Like Malaysia, 
Bangladesh is also a rapidly growing economy with 160 million
population. IT sector is considered a priority sector in Bangladesh over the
last decade. The software development industry dominates this sector. According to the
Bangladesh Association of Software and Information Services (BASIS), 1100+
software companies operate in Bangladesh, where around 40\% have a global
business. The foreign revenue earned by the industry is over 800 Million
USD~\citep{BASIS2018}. Therefore, an understanding of the development practices in Bangladesh and a comparison of such practices 
with other countries can inform us of the issues and characteristic of an emerging software industry.
% In particular, our aim is to figure out the similarities and dissimilarities
% in the development practices of Bangladesh with other countries, which would
% eventually reveal the areas where this industry needs improvement in such emerging countries.
 
% The purpose of this research is to systematically study and characterize the Software
%  Engineering (SE) practices in Bangladesh, i.e., the tools and technology used
%  for design, development, testing, and deployment. Similar studies were carried
%  out in other countries such as Canada, Turkey, Netherlands, and New Zealand. The
%  current study thus also situates a comparative picture of SE practices in Bangladesh
%  and other countries. In particular, our aim is to figure out the similarities and dissimilarities
% in the development practices of Bangladesh with other countries, which would
% eventually reveal the areas where this industry needs improvement in such emerging countries.
 
Our study has four steps. First, we conduct a series of semi-structured interviews of eight software professionals from four 
leading software companies in Bangladesh. We corroborate the interview findings with findings from similar studies of 
other countries like 
Canada, Turkey, Netherlands, New Zealand, etc.~\citep{Garousi2013, Garousi2015, Vonken2012, Wang2018}. 
The purpose is to gain an overview of the overall development practices and challenges. 
Second, we design a survey based on the insights gained from the interviews. A total of 137 
software practitioners from diverse software companies in Bangladesh responded to the survey. Third, 
we analyzed the survey responses to understand the software development practices and challenges in Bangladesh. Fourth, 
we compare our findings against findings from other countries. We answer two research questions:
\begin{inparaenum}[(RQ1)]
  \item What are the software development practices in an emerging country like Bangladesh?
  \item How do the development practices in Bangladesh differ from other countries?
\end{inparaenum} We report the practices (RQ1) and the comparisons (RQ2) along four dimensions (D) as follows. The four dimensions 
were previously used in similar country-based previous studies, which thus helped us to compare our findings against their findings.

%\begin{description}
%\item[\bf{D1. Software development methodologies used.}] 
\nd\bf{D1. Software development methodologies used.} We investigate the development
approaches, methodologies, and requirements analysis processes. We find that the software
companies in Bangladesh mostly follow the agile methodology. In comparison with the
other countries, Bangladeshi software companies generally spend
more time on the implementation stage of the development. Technologically advanced countries spend more on system design.

%\item[\bf{D2. Software tools and techniques used}.] 
\nd\bf{D2. Software tools and techniques used}. We determine the
trending technologies like technology platforms, programming languages,
frameworks, etc. We find that web-based software services
are prevalent in the Bangladeshi software market and JavaScript is mostly used language for web
development. The degree of technologies and tools usage varies
from the developed countries due to the availability of experienced developers,
budget, etc. 
%\item[\bf{D3. Software testing and devops practices used}.] 

\nd\bf{D3. Software testing and devops practices used}. We explore the
present situation of testing and deployment practices adopted by the software
firms in Bangladesh by asking questions about testing and deployment tools, test
automation level, version control system, etc. We find
that the usage of test automation and deployment tools is not widespread in the
Bangladeshi industry. We also see that compared
to developed countries, the test automation tool adoption is inadequate. 
%\item[\bf{D4. Security and performance measures used}.] 

\nd\bf{D4. Performance and security measures used}. We analyze how
software companies secure and maintain their code and what practices are
followed to ensure performance and scalability. The responses show that
standards are followed for security assurance, tools are used for performance
testing, and scalability is mostly ensured by using cloud services. However, the companies in Bangladesh lags
behind in the area of automated performance testing and containerization that
might ensure resource-optimized scalability.
%\end{description}

% \begin{itemize}[leftmargin=10pt]
% 
%     \item \textbf{Software methodologies used in your project?}
%     
%     We investigate the development approaches and methodologies, requirements analysis processes, etc. It helps us to analyze the dominant practice of the industry and found that the software companies mostly follow the agile development model. In comparison with the other countries, it is found that Bangladeshi software companies generally spend significant time on the implementation stage of the development whereas technologically advanced countries spend more on system design and planning.
%         
%     \item \textbf{Which implementation technologies and tools are adopted by software development professionals?}
%         
%     This is to find out the current trending technologies like technology platforms, programming languages, frameworks, etc., in the software industries of Bangladesh that may help one who wants to pursue his career here. We have found that web-based software services are prevalent in the market and JavaScript is mostly used language for web development. The degree of use of different technologies and tools usually vary from country to country due to the availability of experienced developers, budget, etc. Hence, we are interested to explore if the companies in Bangladesh deviate from world standards for particular application domains.
% 
% %\anindya{One line result summery is there for last 2 RQs. For consistency, something should be mentioned for the first two as well.}\khalid{added}        
%         
%     \item \textbf{What type of testing and deployment practices are used?}
%         
%     We want to explore the present situation of testing and deployment practices adopted by the software firms in Bangladesh by investigating testing and deployment tools, test automation level, version control system, etc. This investigation enables us to find out if there is a need for improvement in these areas. We have observed that using test automation and deployment tools is not widespread in the industry, but there is scope of improvement in others. We also see that compared to developed countries, the test automation tool adoption is inadequate. This finding is expected to motivate practitioners to use automated tools for testing.
%         
%     \item \textbf{How are the security and performance ensured in a product of a company?}
%         
%     We try to understand how software companies secure and maintain their code and what practices are followed to ensure performance and scalability. The responses show that standards are followed for security assurance, tools are used for performance testing, and scalability is mostly ensured by using cloud services. In the comparative analysis with the reputed industry, the companies in Bangladesh lags behind in the area of automated performance testing and containerization that might ensure resource-optimized scalability.     
%    
%     %\item \textbf{What software methodology are used in your project?}
%     
%   %  \anindya{The above two comments do not make much sense to me. Please rephrase.}\khalid{updated}
%     
%  %   \item \textbf{Which implementation technologies and tools are adopted by software development professionals?}
%      %   \item \textbf{What types of testing and deployment practices are used?}
%     
%     %\item \textbf{How are the security and performance ensured in the products of a company?}
%     
% \end{itemize}



%\anindya{Gias bhai suggests that we should say that we have 2 research questions. We explore these two from 4 dimensions. In this way we can make the following discussion more precise.}

%More specifically, we answer the following research questions around the two studies:
%We studied from two perspectives: (1) Understanding development practices in Bangladesh: We aim to perceive what methodologies, technologies, and testing practices are followed in software companies. (2) Comparison among the countries: We aim to figure out both similarities and dissimilarities in the development practices of Bangladesh with other countries, which would eventually reveal the areas where this industry needs improvement.
% We want to explore two research questions in this study:
% 
% \begin{itemize}[leftmargin=10pt]
%  
%  \item \textbf{What are the major characteristics of software development practices and methods in Bangladesh?}
%     
%  \item \textbf{What are the similarities and differences in software development practices between Bangladesh and other countries?}
% 
% \end{itemize}

%We explore these research questions from the following 4 dimensions.


% Our findings show that the software companies in Bangladesh usually spend most
% of the time on the implementation stage rather than system design and
% requirement analysis. Also, the use of automated testing and tools for automatic
% deployment is quite low though nowadays, it is an integral part of a better and
% standard software development practice across the globe. Besides, we compare the
% development processes of Bangladesh with other countries and analyze them from
% several perspectives. 
The findings can help software practitioners to
improve development practices in Bangladesh, which then can be applied to other developing countries with similar 
characteristics. The comparison with the developed countries shows the particular avenues for improvement in the emerging countries. 
We also notice the popularity of using cloud services across the software industries in Bangladesh. 
While this is not surprising, it nevertheless can motivate cloud providers to invest in usable and 
readily available automated software design and testing tools into the cloud to cater to the needs of 
emerging software industries. Finally, our study catalogs the
technologies and tools currently being used in Bangladesh's software firms, which could be 
useful for current students to prepare themselves for a career in this emerging software industry.

% For a software development company, the choice of technology evolves over time
% and varies widely according to application type. This study is likely to benefit
% different stakeholders such as potential local and global clients, students,
% i.e., future practitioners, etc. by presenting relevant industry practice.
% Another major benefit of such a study is having insight into different
% dimensions of industrial practice that help design better curriculum for current
% students and reveal the need for continuous education components for existing
% practitioners.
% 
% This study aims to consider detailed activities beyond high-level practices
% gathered through surveying individual professionals. The survey was conducted in
% two phases, a limited primary survey with selected respondents and a final
% survey. The goal of the primary survey was to find any discrepancy or ambiguity
% in the survey questions. Also, the primary survey had an interview session with
% each respondent. From the findings of the interviews, the question of the survey
% was later adjusted. The survey is designed to reveal the methodology of software
% development practices, the adoption of technologies and tools by the
% professionals, and the use of testing and deployment practices in the company
% where the individual concerned works.


\noindent\textbf{Paper Organization.} Section~\ref{related_works} presents the related work to our study.
Section~\ref{study_setup} describes the background of our study and the data
collection procedure. Section~\ref{study_results} answers the two research questions. Section~\ref{discussions} reports
challenges and opportunities as well as analysis by several dimensions of
the findings. Section~\ref{implications} discusses the
implications of our findings. Section~\ref{validity} discusses the threats to
validity. Section~\ref{conclusion} concludes the paper.
