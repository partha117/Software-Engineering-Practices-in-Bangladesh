\subsection{Analysis by Experience}\label{analyze_by_experience}
% see Section 7.3 of \url{https://ieeexplore.ieee.org/abstract/document/8658125}

We highlight the results of our survey's closed questions by the reported professional experiences of the survey respondents in Tables \ref{table:analysis_by_experience_part1} and \ref{table:analysis_by_experience_part2}. The similar principles discussed in Section~\ref{analyze_by_professions} are applied for this report.

% \newcolumntype{b}{X}
\newcolumntype{v}{>{\hsize=.04\hsize}X}
\newcolumntype{k}{>{\hsize=.96\hsize}X}
\newcolumntype{m}{>{\hsize=.2\hsize}X}
\newcolumntype{y}{>{\hsize=.33\hsize}X}
\begin{table}[!ht]
    \centering
    \caption{Highlights of Findings from Survey Closed Questions by Experience}
    \begin{tabularx}{\textwidth}{v|k}
        \hline
        \textbf{No}     & \textbf{Question}  \\ \hline
        5 & What is your current role?\newline 1) less than 2: Developer (91.11\%) 2) 2 to 5: Developer (75.0\%) 3) 5 to 10: Developer (60.71\%) 4) more than 10: Manager (42.42\%) \& Developer (42.42\%) \\ \hline
        
        6 & Which of the following software development methodologies do you follow?\newline 
        1) less than 2: Agile (55.74\%) 2) 2 to 5: Scrum (34.78\%) 3) 5 to 10: Agile (50.0\%) 4) more than 10: Agile (33.33\%) \\
        % {
        % \begin{tabularx}{0.92\textwidth}{yyy}
        %     & top1 (\%) & top2 (\%) \\
        % less than 2 & Agile (55.74)  & Scrum (27.87)  \\
        % 2 to 5 & Scrum (34.78)  & Agile (30.43)  \\
        % 5 to 10 & Agile (50.0)  & Scrum (30.0)  \\
        % more than 10 & Agile (33.33)  & Scrum (31.67)  \\
        % \end{tabularx}
        % } \\
        \hline
        
        7 & Which of the followings do you use for requirements gathering?\newline
        1) less than 2: Plain Text (30.88\%) 2) 2 to 5: Plain Text (29.03\%) 3) 5 to 10: Story board (23.73\%) 4) more than 10: Story board (27.63\%) \\
        % GUI prototype = GP, Grooming seasons = GS, Plain Text = PT, Story board = SB, Use Case = UC
        % {
        % \begin{tabularx}{0.92\textwidth}{yyy}
        %     & top1 (\%) & top2 (\%) \\
        % less than 2 & PT (30.88)  & UC (20.59)  \\
        % 2 to 5 & PT (29.03)  & GP (20.97)  \\
        % 5 to 10 & SB (23.73)  & GS (20.34)  \\
        % more than 10 & SB (27.63)  & UC (19.74)  \\
        % \end{tabularx}
        % }   \\ 
        \hline
        
        8 & On which software development activities, do you spend most of the time?\newline Implementation = Imp, Requirement Analysis = RA
        {
        \begin{tabularx}{0.92\textwidth}{yyyy}
            & top1 (\%) & top2 (\%) & top1 / top2 \\
        less than 2 & Imp (29.9)  & RA (19.59) & 1.53 \\
        2 to 5 & Imp (32.88)  & RA (15.07) & 2.18  \\
        5 to 10 & Imp (21.69)  & RA (19.28) & 1.125 \\
        more than 10 & Imp (23.68)  & RA (23.68) & 1 \\
        \end{tabularx}
        }\\ \hline
        
        9 & Which of the following technologies do you have experience working in?\newline
        1) less than 2: Web (51.43\%) 2) 2 to 5: Web (56.0\%) 3) 5 to 10: Web (46.0\%) 4) more than 10: Web (35.38\%)
        % {
        % \begin{tabularx}{0.92\textwidth}{yyy}
        %     & top1 (\%) & top2 (\%) \\
        % less than 2 & Web (51.43)  & Mobile (24.29)  \\
        % 2 to 5 & Web (56.0)  & Mobile (26.0)  \\
        % 5 to 10 & Web (46.0)  & Mobile (30.0)  \\
        % more than 10 & Web (35.38)  & Mobile (29.23)  \\
        % \end{tabularx}
        % }\\ 
        \\ \hline
        
        10 & What is the primary operating system you are developing on?\newline
        1) less than 2: Windows (46.15\%) \& Linux (30.77\%), 2) 2 to 5: Linux (47.73\%) \& Windows (38.64\%), 3) 5 to 10: Linux (41.46\%) \& Windows (26.83\%), 4) more than 10: Linux (52.27\%) \& MacOS (25.0\%)
        % {
        % \begin{tabularx}{0.92\textwidth}{yyy}
        %     & top1 (\%) & top2 (\%) \\
        % less than 2 & Windows (46.15)  & Linux (30.77)  \\
        % 2 to 5 & Linux (47.73)  & Windows (38.64)  \\
        % 5 to 10 & Linux (41.46)  & Windows (26.83)  \\
        % more than 10 & Linux (52.27)  & MacOS (25.0)  \\
        % \end{tabularx}
        % }\\ 
        \\ \hline
    \end{tabularx}
    \label{table:analysis_by_experience_part1}
\end{table}
\newcolumntype{b}{X}
\newcolumntype{v}{>{\hsize=.03\hsize}X}
\newcolumntype{m}{>{\hsize=.2\hsize}X}
\newcolumntype{y}{>{\hsize=.33\hsize}X}
\begin{table}[htbp]
    \centering
    \caption{Highlights of Findings from Survey Closed Questions by Experience}
    \begin{tabularx}{\textwidth}{v|b}
        \hline
        \textbf{No}     & \textbf{Question}  \\ \hline
        
        11 & Which programming languages are you using?\newline
        {
        \begin{tabularx}{0.92\textwidth}{yyy}
         & top1 (\%) & top2 (\%) \\
        less than 2 & Java (31.11)  & JavaScript (30.0)  \\
        2 to 5 & Java (28.57)  & JavaScript (28.57)  \\
        5 to 10 & JavaScript (28.17)  & Java (23.94)  \\
        more than 10 & JavaScript (27.03)  & Java (24.32)  \\
        \end{tabularx}
        }\\ \hline
        
        12 & Which frameworks are you using? \newline
        {
        \begin{tabularx}{0.92\textwidth}{yyy}
         & top1 (\%) & top2 (\%) \\
        less than 2 & Spring (35.56)  & Django (13.33)  \\
        2 to 5 & Others (25.58)  & Spring (18.6)  \\
        5 to 10 & Spring (29.55)  & Laravel (15.91)  \\
        more than 10 & Spring (27.78)  & Others (14.81)  \\
        \end{tabularx}
        }\\ \hline
        
        13 & What types of software testing practices do you use? \newline Functional testing = FT, Performance testing = PT, Unit testing = UT, User acceptance testing = UAT
        {
        \begin{tabularx}{0.92\textwidth}{yyy}
         & top1 (\%) & top2 (\%) \\
        less than 2 & UT (32.1)  & FT (24.69)  \\
        2 to 5 & FT (24.64)  & UT (18.84)  \\
        5 to 10 & FT (23.44)  & UAT (20.31)  \\
        more than 10 & UT (25.29)  & UAT (19.54)  \\
        \end{tabularx}
        }\\ \hline
        
        14 & What is the level of automated testing in your projects?\newline 
        1) less than 2: \textbf{\textit{5.0, 34.88\% } }
        2) 2 to 5: \textbf{\textit{5.0, 35.71\% } } 3) 5 to 10: \textbf{\textit{5.0, 30.77\% } } 4) more than 10: \textbf{\textit{2.0, 27.27\% } }
        \\ \hline
        
        15 & Which tools do you use for testing and quality assurance?
        {
        \begin{tabularx}{0.92\textwidth}{yyy}
         & top1 (\%) & top2 (\%) \\
         & top1 (\%) & top2 (\%) \\
        less than 2 & XUnit (33.33)  & Selenium (26.67)  \\
        2 to 5 & Selenium (45.45)  & XUnit (27.27)  \\
        5 to 10 & Selenium (35.48)  & XUnit (32.26)  \\
        more than 10 & XUnit (38.46)  & JenKins (25.64)  \\
        \end{tabularx}
        } \\ \hline
        
        16 & Which tools do you use for continous deployment? \newline
        AWS codeDeploy = AC, Open Source server = OSS
        {
        \begin{tabularx}{0.92\textwidth}{yyy}
         & top1 (\%) & top2 (\%) \\
        less than 2 & AC (23.53)  & Others (17.65)  \\
        2 to 5 & AC (26.67)  & Bamboo (20.0)  \\
        5 to 10 & AC (27.27)  & Others (27.27)  \\
        more than 10 & Jenkins (37.5)  & AC (16.67)  \\
        \end{tabularx}
        } \\ \hline
    \end{tabularx}
    \label{table:analysis_by_experience_part2}
\end{table}

With the increase of professional experience, we see the percentage of software
developer being decreased but manager being increased according to Q5 in
Table~\ref{table:analysis_by_experience_part1}. On top of that, from Q7, we see
to gather requirements of a software project, employees up-to mid-senior level,
most of whom are developers, tend to use plain text, where more than 5 years
experienced professionals prefer storyboard.

From the Q8 in Table~\ref{table:analysis_by_experience_part1}, we see
\bf{implementation among all other development activities being the main concern for
all levels of experienced employees. However, the ratio of the top two
activities points out that the more senior an employee is, the more he/she tends
to analyze the requirements of a software project.}

As per the Q10 in Table~\ref{table:analysis_by_experience_part1}, we find that at the initial stage of career, professionals are inclined to prefer Windows most and then they mostly use Linux in mid-career and gradually they tend to use macOS at late-career. It might indicate that employees were proficient in Windows most before the start of their career. In Q5 of as per the Q10 in Table~\ref{table:analysis_by_experience_part1}, we see the percentage of developer up-to mid-career is the most and in late-career, the percentage of managers is noticeable which point out that developers are inclined to use Windows and Linux, and managers prefer macOS for their managerial tasks.

When participants were asked about their software testing practices, most of them per experience level were concerned about the unit and functional testing, pictured in Q14 of Table~\ref{table:analysis_by_experience_part2}. Moreover, we have found that senior participants are more prone to making software products accepted among clients using user acceptance testing (UAT). However, if we looked at Q15 when asked to level the automatic testing of their projects, the percentage is gradually decreasing with the seniority level like the ratio of the development activities of implementation and requirement analysis discussed earlier in this section. And most of the more than ten years of experience participants leveled test automation `2' for their projects. It implies that when it's time for testing practices like user acceptance testing rather than unit and functional testing, software industries are less likely to use test automation tools in their projects.
\bf{Testing practices like unit testing, functional testing, etc., used for implementation purposes, are mainly carried out as test automation in Bangladesh. But, for other testing practices like user acceptance testing, software companies tend not to carry out test automation tools more often.}

At Q17 of Table \ref{table:analysis_by_experience_part2} we observe that using tools for continuous deployment is commensurate to the years of professional experience. Personnel of senior-level are more likely to work in development and operations (DevOps) related fields.