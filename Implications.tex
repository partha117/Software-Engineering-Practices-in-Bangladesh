\section{Implications of Findings}

% what are the implications of findings? say for another emerging country or software professionals or companies like Microsoft, Google?

\label{implications}
\input{Challenges_and_Opportunities}
\subsection{Implication to Stakeholders}
Thus far, we have discussed the SE practices of Bangladesh and also their similarities/dissimilarities with both established and emerging SE industries. In this section, we discuss the implications of our findings. Apart from helping the researchers to understand SE challenges in the emerging software industry, our study can also help industry leaders, software practitioners, universities, and open source development projects.

\indent \textbf{Researchers:} In this study, we have identified that the participation of female in the Bangladesh SE industry is comparatively less than that of male engineers. \anindya{From which part of the study you found this? No RQ has any result on it as far as I remember.}\partha{updated}\partha{Based on section \ref{analysis by gender}} Hussain et al.\citep{Hussain2020} expressed a concern that there may be bias in the hiring process of the industry. Further research may be conducted to identify the inherent reason behind these facts. We have also found that testing has been given less priority in the industry though it is one of the key elements for ensuring software quality. Furthermore, automatic testing, which is crucial for the efficiency and productivity of the test team, is rarely used. An automatic testing framework for other types of testing (such as GUI testing) may increase automatic testing practice. The price of quality automated testing tools may cause a lack of their use as the software budget in developing countries is often limited. Researchers may create a user-friendly framework to implement automatic testing for different types of testing to address this affordability issue. This finding may encourage researchers to develop open-source tools for the industry of developing countries. 

% We have found that certain development methods are popular with a particular group (agile in females). Further study may be conducted to investigate the problems associated with incorporating other development methods in those groups. \anindya{Development methods are determined by the technical management of a company. Correlating it with the female/male respondent is not a practical idea, I think. This point may be omitted.}

Containerization technologies are a popular trend worldwide. However, the practice of containerization technologies is too low. Automatic deployment/release is another under-practiced area. Researchers can investigate the challenges to incorporate these technologies in an emerging SE industry like Bangladesh.

\indent \textbf{Software Practitioners:} We have found that rigorous testing practice is not prevalent in Bangladesh. The difference in testing effort between the established software industry (e.g., Canada) and Bangladesh is too high. The scenario is also true for other developing countries like Pakistan. From our study, software practitioners may have an idea of their standing in software QA in testing. Similar to all emerging industries, security testing is less prioritized in Bangladesh. From our comparison, developers may better understand security testing and security practices in other countries. Moreover, we have shown the level of automatic testing in the Bangladesh SE industry. Automatic testing is directly related to productivity. Practitioners can increase their productivity by implementing automatic testing for their products. Similarly, we have observed the under-use of automatic deployment/CI-CD tools. Practitioners should focus more on this area to automate their product pipelines.

\indent \textbf{Universities:} We have found that certain languages (e.g., Java, JavaScript, etc.) and frameworks (e.g., Spring, Django, ASP.NET) have extensive use in the software industry of Bangladesh. Universities can update their curricula to meet industry demand. The students aspiring to join the software industry must prepare themselves accordingly to be productive quickly. We have also found less automated testing practices in the industry of Bangladesh. This may be related to a lack of exposure to the testing framework. Similar to Hussain et al.\citep{Hussain2020}, we also suggest including testing-related courses into the curriculum of universities and introducing relevant assignments to have hands-on experience on automated testing tools right from the student level. The students in their academic projects should also use container and other DevOps tools to bring some qualitative improvement for the industry when they join there.


\indent \textbf{Industry Leaders:} This study found that the Bangladesh software industry lags in adopting some of the current industry trends. Bangladesh can be a good marketplace for cloud companies like Microsoft, Google, and Amazon if they provide an affordable package for the software companies operating here. Local entrepreneurs may also think of building cost-effective local public clouds along with providing some common DevOps services.  

Moreover, we have found most of the participants in our study are engaged in web technology. This may indicate that the web is the most popular form among users of Bangladesh. This observation can help the industry owners to select appropriate medium while targeting users of this region. %We have identified common practices of the practitioners of the Bangladesh software industry.
