\subsubsection{Development methods and practices}
\label{dev_methods_comparison}

To find out the overall comparison of this section, we report the following sub-sections:

\begin{itemize}
\item Software development methodologies (Q 6).
\item Requirements gathering (Q 7).
\item Most time consuming software development activities (Q 8).
\end{itemize}


\paragraph{Software development methodologies}
From our study we see that the most acceptable model, that is regularly and always used, is the agile model (64\%) in Bangladesh but the usage of the scrum (44\%) in New Zealand has greater usage followed by agile (30\%) \cite{Wang2018} and in Turkey, waterfall is mostly used based on the earlier 2015 survey \cite{Garousi2015}. Again, in both Bangladesh and New Zealand, extreme programming (XP) has a lower percentage of usage. Almomani el al. \cite{Almomani2015} found that software developments in Malaysia are predominantly regulated through ad-hoc approach (53\%) and the agile methodologies (46\%) since usually software organizations are majorly concerned with short-term delivery of software products.


\paragraph{Requirements Gathering}
According to \ref{fig:requirements}, using plain text (44\%) and story board (41\%) are the most widely used requirements gathering. This result is similar with the survey of Vonken et al. \cite{Vonken2012}. From their study we can find that the textual description of specifying requirements is a firm favourite in Netherlands.


\paragraph{Development activities timeline}
According to the study of Wang et al. \cite{Wang2018}, during system design and development, most time is spent on implementation and coding and also relatively less time is spent on maintenance in New Zealand which is similar with Bangladesh based on our study. But requirement analysis, the activity, requires the second most time to spend in Bangladesh according to 45\% respondents of our survey where in Malaysia, as per \cite{Baharom2006}, most organizations spend from 5\% to 20\% of their efforts for planning and requirement analysis. Again, from the study of Cusumano et al. \cite{Cusumano2003}, architectural, functional, and design specification documents are mostly used, popular and well-regarded practice rather than just writing code with minimal planning and documentation in India, Japan, and Europe.
