\section{Related Works}

There are many works about software development procedures outside of our country, and those researches were based on that particular country. Nevertheless, we have come to know a great extent of knowledge from those researches.

In 2012, Vonken et al. \citep{Vonken2012} conducted a survey among Dutch software producing organizations to determine whether there is a gap between the current state of the practice and state of the art in software engineering. From 99 respondents, they extracted 22 interesting observations. These observations mark insights into the development process that they found unusual, or surprising, at least from an academic point of view. This unusualness could either stem from certain principles being applied less or more frequently than one would expect, or from unexpected correlations that were observed between factors.

The survey in \citep{Garousi2015} studies the software practices in Turkey to characterize and understand the state of its SE practices. The military and defense software sectors are quite prominent in Turkey, especially in the capital Ankara region, and many SE practitioners work for those companies. 54\% of the participants reported not using any software size measurement methods, while 33\% mentioned that they have measured lines of code (LOC). In terms of effort, after the development phase (on average, 31\% of overall project effort), software testing, requirements, design and maintenance phases come next and have similar average values (14\%, 12\%, 12\% and 11\% respectively). Respondents experience the most challenge in the requirements phase. Waterfall, as a rather old but still widely used life-cycle model, is the model that more than half of the respondents (53\%) use. The next most preferred life-cycle models are incremental and Agile/lean development models with usage rates of 38\% and 34\%, respectively. The Waterfall and Agile methodologies have slight negative correlations, denoting that if one is used in a company, the other will less likely to be used.

A recent survey conducted by Wang et. al. \citep{Wang2018} in 2018 shows that, New Zealand professionals use similar methodologies as professionals in other countries. Popular programming languages differ somewhat to popular languages in other rankings. Quality assurance is rather ad-hoc and the release process is inspired by agile software development principles. Our findings highlight some differences of the New Zealand software industry to other countries. Furthermore, we identified some strengths and weaknesses related to processes and practices. Our findings can help software professionals and organizations reflect on (and potentially adjust) the way they work.

From another study \citep{Groves2000}, we report on the software development techniques used in the New Zealand software industry, paying particular attention to requirements gathering. We surveyed a selection of software companies with a general questionnaire and then conducted in-depth interviews with four companies. Our results show a wide variety in the kinds of companies undertaking software development, employing a wide range of software development techniques. Although our data are not sufficiently detailed to draw statistically significant conclusions, it appears that larger software development groups typically have more well-defined software development processes, spend proportionally more time on requirements gathering, and follow more rigorous testing regimes.

The study conducted by Sison et. al. \citep{Sison2006} presents exploratory survey and case study results on software practices of some software firms in five ASEAN countries (Malaysia, Philippines, Singapore, Thailand and Vietnam), and provides directions for further research on software practices in the ASEAN/Southeast Asian region.