\subsubsection{What type of testing and deployment practices are used?}

To get the picture of quality assurance (QA) in countries like Malaysia, Turkey, France etc., different studies have been published. In Table \ref{table:testing_comparison}, we compare prime outcomes on QA topics of our study with the outcomes of previous studies. We will compare the following aspects of QA among those countries to get a relative picture of QA in Bangladesh:

\begin{itemize}
    % \item Requirements Clarity
    \item Software Testing Practices (Q 14)
    \item Level of Automated Testing (Q 15)
\end{itemize}

\begin{table}[]
\caption{Comparison of testing practices and test automation between our findings and prior findings}

\begin{tabular}{llll}

\hline
\multicolumn{1}{c}{\textbf{Practices}} & \multicolumn{1}{c}{\textbf{Our Study}} & \multicolumn{1}{c}{\textbf{Prior Study}} & \multicolumn{1}{c}{\textbf{Comparison}} \\ 
\hline 


\multicolumn{1}{l|}{\multirow{2}{*}{\parbox{0.1\textwidth}{What type of testing and deployment practices are used?}}
} 
& 
\multicolumn{1}{l|}{\multirow{2}{*}{\parbox{0.25\textwidth}{
\vspace{-50pt} (1) Around half of our respondents responded about carrying out unit testing and functional testing. Also, the utilization rate of acceptance and UI testing has an appreciable percentage in Bangladesh. (2) Automated testing exercise is not usual in Bangladesh as per the majority of our respondents.
}}
} 
& 
\multicolumn{1}{l|}{\comparisoncell{0.25}{\vspace{50pt}Unit testing is observed to be the most exercised in Malaysia (68.29\%) \citep{Baharom2006}, Canada (79.27\%) \citep{Garousi2013}, New Zealand (73\%) \citep{Wang2018} in a great percentage.
}}                                  

&

\multirow{2}{*}{\parbox{0.23\textwidth}{
\vspace{-70pt} We have perceived that unit testing is moderately practiced in Bangladesh, though its usage is comparatively spacious in other countries. In the adoption of test automation, the software industry of Bangladesh is way behind France though the usage rate might have similar to other European countries.
}} \\ \cline{3-3}

\multicolumn{1}{l|}{}                                       
& 
\multicolumn{1}{l|}{}                                       
& 

\multicolumn{1}{l|}{\comparisoncell{0.25}{
\vspace{50pt} Test automation is highly embraced by France software industry and comparatively less adopted in overall Europe~\citep{dutta1999}.
}} 
&                                         
\\ \hline

\end{tabular}
\label{table:testing_comparison}
\end{table}



\paragraph{Software Testing Practices}
From our study, as per \ref{fig:testing}, we see an interesting point that unit testing (53\%) and functional testing (49\%) are moderately used in Bangladesh where from \cite{Wang2018} and \cite{Garousi2013} we can see that a high percentage of respondents in both Turkey and Canada rely on unit testing. On the other hand, usage rate of both acceptance testing and UI testing is quite similar in these countries. In Malaysia, based on \cite{Baharom2006}, Baharom et al. reported that unit testing, integration testing, system testing and acceptance testing are used by most of the organizations in high percentage and about half of the organizations is carrying out alpha and beta testing.

\boxtext{Software engineers around the world usually give unit testing first priority.}


\paragraph{Level of Automated Testing}
We have found that, as per \ref{fig:autoTest}, around 25\% of our respondents are highly concerned with automated testing while around 35\% of our respondents have chosen as medium level concern and rests are barely using automated testing. Corresponding to the study \cite{dutta1999}, usage rate of automation testing tools in overall Europe is 26\% where in France, being highly adopted country, this is 61\% but in Israel this rate is only 9\%.
