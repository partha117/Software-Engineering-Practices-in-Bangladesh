\begin{abstract}
\hfill
\newline
\noindent\textbf{Context:}
%\anindya{Changed the context. Please check.}
Software development is complicated with rapidly changing requirements, techniques, processes and 
the involvement of diverse stakeholders (clients, developers, testers, managements, and so on). 
It is often the case a burgeoning community shapes around the emergence of a software industry 
in a location (e.g., silicon valley) or in a country. Therefore, 
a study on the software development practice in the industry of a country
reveals the challenges and scopes of improvement for that industry and
industries of similar types. There are several studies on the software industry
of developed countries. However,
to the best of our knowledge, no recent study explores the issues
of an emerging software industry. 
% The software engineering industry in Bangladesh offers a rapidly emerging
% environment and significantly contributes to the national economy. A thorough
% study to understand the software development practices is essential to figure
% out the challenges and opportunities to deliver high-quality software.
\newline \textbf{Objective:} In this research, we focus on understanding the challenges and opportunities within the software companies of Bangladesh, which is an emerging developing country with a large group of skilled software practitioners. We have two objectives. First, we aim to understand the methods and practices used
in the dynamic environment of software development. This analysis informs us of the practices adopted and challenges faced by the software companies in an emerging country. Second, we aim  to understand whether and how the development practices 
and methods in Bangladesh differ from other countries. The comparison informs us of the 
uniqueness of software industries in emerging countries like Bangladesh. 
\newline \textbf{Method:} Our study has two phases. First, we conduct a series of semi-structured interviews with eight individual practitioners from four software companies. The purpose is to understand the overall development practices and methods. Second, we use the insights gained from the first phase to design a survey. A total of 137 software practitioners from a diverse set of companies responded to the survey.
\newline \textbf{Results:} The major findings of our study on the companies of
Bangladesh software industry are: i) Most companies follow agile methodology
and give implementation-related activities a high priority, ii) Most companies
are used to developing web-based software products, iii) Most companies use less
automated testing and deployment tools, and iv) Several companies seriously
consider the security, performance, and scalability of their software.
\newline \textbf{Conclusion:} A comparison of the findings in Bangladesh across other countries shows that 
the development practices are similar to other development countries (e.g., Malaysia), but differ from 
those in developed countries (e.g., Japan). For example, due to the mostly outsourcing nature of work carried in emerging software 
industries like those in Bangladesh, overall system design is not as highly practised as the industries in developed countries like Japan (or Europe). 
Automated testing is an area where these emerging companies also need to catch up with the developed countries. 
In contrast, critical non-functional requirements like security, scalability are equally considered (and practiced) as important 
across all regions. The findings offer room for improvements for emerging software industries. More emphasis can 
be placed upon promoting overall system design and automated testing in emerging software industries. 
\end{abstract}