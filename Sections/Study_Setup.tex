\section{Study Setup}
\label{study_setup}
% \section{Study Setup}
% overview
% \subsection{Participants}
% how the participants are selected, what is the response rate? 
% any incentives were provided to encourage participation?
% \subsection{Data Collection and Analysis}
% how was the data collected (Google form), how was the data analyzed (open coding for open-ended questions, statistical analysis for closed questions) see \url{https://ieeexplore.ieee.org/abstract/document/8658125}

For this study, we conducted an interview session and a survey. The goal of the interview session was to reduce ambiguity from the survey questions. In total, eight individual participants from four organizations were randomly selected for the interview session. In the session, participants first completed the survey, and then we interviewed participants to identify ambiguities in the question. From the feedback and results of the interview session, we revised the questions. The original survey was conducted through google form.

\subsection{Participants}
\label{survey_participants}
In this survey, we targeted developers who are currently working in the Bangladesh software industry. We applied purposive sampling\cite{Vogt2005} to include respondents in a software development related role. We shared the survey link through the authors' personal connection and in the local developers' groups on social media to achieve our sampling goal. We also implemented the chain referral strategy\cite{creswell2013} and asked others to pass on the survey invite. Following this strategy, we reached out to an unknown number of potential respondents. Thus we can calculate the response rate of our survey. In total, We have received 137 responses from the survey. The current role of the participants is (1) Software Developer 64.24\% (2) Manager 15.89\% (3) SQA Engineer 6.62\% (4) Business Analyst 1.32\% (5) R\&D Engineer 1.32\% (6) Software Engineer 1.32\% (7) Role not disclosed 5.96\% (8) Data Engineer 0.66\% (9) Software Architect 0.66\% (10) Team Lead 0.66\% (11) Trainer 0.66\% (12) UX Designer 0.66\%. The distribution of experience of the participants is (1)Less than 2 years 33.58\% (2) 2 to 5 years 24.82\% (3)5 to 10 years 18.98\% (4) more than 10 years 17.52\% (5) experience not disclosed 5.11\%. Therefore, at least 61.31\% respondents of our survey has worked in the industry for at least 2 years.



\subsection{Data Collection \& Analysis}
\label{survey_data_collection}

We conducted the survey through google form. The survey link was opened before the invitations were sent, and the survey link was closed for two consecutive weeks without any response. The survey link was open for feedback for about two months. 

A systematic qualitative data analysis process was followed to analyze the open-ended questions. First, the two authors independently coded the first 30\% of each question's responses to extract potential categories. Second, using these categories, the authors conducted discussion sessions to develop a unified common coding scheme for each question. Third, the rest of the responses were coded using this coding scheme using the Coding Analysis Toolkit (CAT)\cite{Lu2008} software. We used Cohen's kappa\cite{Cohen1960} to measure the agreement between the author's codes. Kappa value was 0.65. It is a common practice that the kappa value between 0.61 and 0.80\cite{Landis1977} is considered a `substantial agreement’. For statistical analysis, we have used the statsmodels\cite{seabold2010} module in Python.
 
