\section{Related Works}
\label{related_works}
% \section{Related Work}
% \subsection{Studies of Development Practices and Processes}
% any survey that did not focus on specific region but tried to look at the theme in general.
% \subsection{Related Region-specific Studies}
% NZ, Turkey, Canada, etc. any previous study in Bangladesh?
This section presents the related works that focus on software development practices and processes globally and in specific countries and regions.  

\subsection{Studies of Development Practices and Processes}
\label{dev practice study}
Cusumano et al.~\citep{Cusumano2003} have conducted a global survey on a completed software engineering project to identify software engineering practices. They have found that detailed architectural design and documentation is a common practice worldwide except the USA. In the USA, only 32\% of projects used detailed design specifications. One of the interesting findings of their study is the completeness of design before coding negatively correlates with the number of defects.

AlSubaihin et al.~\citep{AlSubaihin2019} have identified the influence of the app store on software engineering practices. They have found that the perception of quality is slightly different among app store developers. App Store developers gave more priority to user rating than the traditional metrics like code quality and documentation when measuring software quality.

To identify the state of the practices in start-up companies, Klotins et al.~\citep{Klotins2018} have conducted a study on start-up companies. They found that start-ups apply market-driven requirements engineering instead of the standard software engineering requirement engineering. However, the applied requirements engineering practices are often rudimentary and lack alignment with other knowledge areas (such as design).

\subsection{Related Region-specific Studies}
\label{region specific study}

In 2012, Vonken et al.~\citep{Vonken2012} surveyed Dutch software producing organizations to determine whether there is a gap between the current state of the practice and state of the art in software engineering. From 99 respondents, they extracted 22 interesting observations. These observations mark insights into the development process that they found unusual or surprising, at least from an academic perspective. This unusualness could either stem from certain principles being applied less or more frequently than expected or from unexpected correlations observed between factors.

The survey conducted by Garousi et al.~\citep{Garousi2015} studies Turkey's software practices to characterize and understand the state of its SE practices. The military and defense software sectors are quite prominent in Turkey, especially in the capital Ankara region, and many SE practitioners work for those companies. 54\% of the participants reported not using any software size measurement methods, while 33\% mentioned that they had measured lines of code (LOC). In terms of effort, after the development phase (on average, 31\% of overall project effort), software testing, requirements, design, and maintenance phases come next and have similar average values (14\%, 12\%,12\%, and 11\% respectively). Respondents experience the most challenge in the requirements phase. As a rather old but still widely used life-cycle model, the Waterfall is the model that more than half of the respondents (53\%) use. The next most preferred life-cycle models are incremental and Agile development models with 38\% and 34\% usage rates, respectively. The Waterfall and Agile methodologies have slight negative correlations, denoting that if one is used in a company, the other will less likely to be used.
A recent survey conducted by Wang et al.~\citep{Wang2018} in 2018 shows that New Zealand professionals use similar methodologies as professionals in other countries. Key findings of their study are, (1) popular programming language in the New Zealand software industry does not match with the worldwide ranking of popular languages, and (2) most of the time in SDLC is spent on implementation-related activities rather than analysis and design.

In another study, Groves et al.~\citep{Groves2000} reported that the New Zealand software industry pays particular attention to requirements gathering. They surveyed a selection of software companies with a general questionnaire and then conducted in-depth interviews with four companies. They found a clear difference in the testing phase between large and small software companies. Their finding is larger companies pay more attention to testing than smaller companies.

The study conducted by Sison et al.~\citep{Sison2006} presents exploratory survey and case study results on software practices of some software firms in five ASEAN countries (Malaysia, Philippines, Singapore, Thailand, and Vietnam). They found that most of the firms in that region do not follow the standard procedure for SQA.
In a study focusing on the test practices in Canadian firms, Garousi et al.~\citep{Garousi2013} found that the number of passing user acceptance tests and the number of defects found per day are considered the most important quality assurance metrics in Canadian firms. They compared their result to a previous study and showed that Canadian firms are giving more importance to testing related training than in the past.

Baharom et al.~\citep{Baharom2006} had conducted a study on Malaysian software firms to find the effectiveness of standard practices. They found that alpha and beta testing was hardly implemented in software firms. Another interesting finding of their study is that most Malaysian firms emphasize implementation; only a negligible number of companies spend more than 20\% of the effort in planning and design.
Zafar et al.~\citep{Zafar2018} have surveyed why Pakistani software firms do not follow the standard requirement engineering process. They have found multiple factors contributing to the issue, such as lack of budget, lack of time, lack of dedicated team, etc. However, the most prevalent issue is lack of budget; more than 60\% of their respondents have responded that the standard requirement engineering process was not followed due to scarcity of budget.

Based on a 200 participants survey, Hussain et al.~\citep{Hussain2020} concluded that computer science undergraduate education system in Bangladesh leaves most of its graduates unprepared for the software industry. They suggested that updating the syllabus as part of the curriculum and including internships could help make graduates fit for the industry.

To identify the software engineering practices in Bangladesh, Rahim et al.~\citep{Rahim2017} have surveyed 41 practitioners in the Bangladesh software industry. One of their interesting findings is the waterfall model is still popular among them. They found that 40\% of respondents indicated that requirement analysis and prioritization are the most challenging software development process. In another survey focusing on testing practices in the Bangladesh software industry, Bhuiyan et al.~\citep{M2018} found that most companies do not follow any standard SQA techniques for their projects. The interesting fact is such malpractice does not hinder their progress; they reported that these firms are in the industry for 6.5 years on average. However, in another survey, Begum et al.~\citep{Begum2009}  found that 47.5\% of respondents follow standard SQA techniques in their projects.