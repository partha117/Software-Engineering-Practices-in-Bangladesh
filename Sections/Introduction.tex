% \section{Introduction}
% we will say there was no previous study on how software industry in an emerging environment works, the challenges they face and the practices they adopt. we will say we fill this gap by taking the software industry in BD as a case study, since BD is a rapidly developing country and part of this rapid development is attributed to its emerging software industry. as such the data from the industry of BD can be insightful for similar other countries as well as for companies who normally look for emerging economies as a place for growth (e.g, Apple and Google both are focusing more and more on India and China for their expansion and revenue increase, because both of those are emerging - although both are much bigger economy than BD, etc.)

% We answer the following research questions:
% \begin{enumerate}[label=RQ\arabic{*}., leftmargin=25pt]
%     \item What development practices and processes are used by professional software developers in Bangladesh?
%     \item How do the development practices and processes differ between Bangladesh and other countries?
% \end{enumerate}



\section{Introduction}
\label{introduction}

The software industry in Bangladesh is considered relatively small compared to the size of its population (160 million plus), and the size of the national economy. Yet, the software industry in this country has started rapidly emerging and become part of Bangladesh's rapid development in recent years. Majority software companies were established after 2000 and have become crucial resource for the economy of Bangladesh \cite{Basis2017}, and also the growth rate is expected to continue. This propitious growth is supported by export trends and large demand for IT automation in domestic market. Thus for this, we have to first point out how software development process is exercised in such a rapidly emerging industry. 

The practice of software development varies from region to region. A certain methodology may become more efficient in some region which is not considered efficient in another regions. Although most of the factors behind software development are same all the time, there are certain points which vary from region to region. For software development professionals, deciding on processes, practices and techniques is critical. Sometimes, these decisions might be based on marketing and literature bias that supports new or industry-supported practice.

Many studies concerning the practices of the software development have been conducted. Those studies convey similarities, but also differences of diverse regions, countries and communities. Previous empirical studies have gone into how software development practices were conducted in the North America, Europe, Turkey, New Zealand etc. There are some previous studies on the Bangladesh software engineering (SE) industry, but the studies' focus and dimension were different. To the best of our knowledge, this is the first study to identify software engineering practices in emerging countries by taking the software industry in Bangladesh as a case study on how software industry in an emerging environment works, the challenges they face and the practices they adopt.

The aspiration of this survey is to extensively look over the methodology of software development practices, the adoption of technologies and tools by professionals, and the use of testing and deployment practices in the software industry of Bangladesh. Also, we are interested in the developers' and managers' opinion about their practices and expectation regarding software development related processes. On this purpose, we categorize our study into 2 categories: (1) Understanding Development Practices in Bangladesh:  We aim to perceive what methodologies, technologies, testing practices software companies use as well as how the security is ensured. (2) Comparison among Countries: We aim to figure out both similarities and dissimilarities in development practices of Bangladesh with other countries to see the characteristics in which Bangladesh is prevalent and in which Bangladesh needs improvement. More specifically, we answer following research question around the two studies:

\begin{itemize}[leftmargin=10pt]

\item \textbf{Study  1. Understanding Development Practices in Bangladesh}: We answer four research questions as follows:
    \begin{enumerate}[label={\textbf{RQ\arabic{*}.}}, leftmargin=30pt]
    
    \item \textbf{What software methodology are used in your project?}
    
    This question investigates the approaches of development by reporting the results of development methodologies, requirements gatherings, and most time consuming activities. These outcomes help us to get the current picture of software development methods in Bangladesh.
        
    \item \textbf{Which implementation technologies and tools are adopted by software development professionals?}
        
    There exists a plethora of technologies around the globe but in using these, it varies countries to countries. So, in this question, we try to find out the current trending technologies like technology platforms, programming languages, frameworks, etc. in software industries of Bangladesh which may help one who wants to pursue his career in Bangladeshi software industry.
        
    \item \textbf{What type of testing and deployment practices are used?}
        
    In this research question, we want to point out the present situation of testing and deployment practices which are adopted by Bangladeshi software firms by investigating several points like testing and deployment tools, test automation level, version control system, etc. This investigation may enable us to find out if there needs any improvement in the part of quality assurance of a software product. We have found that there has a tendency in Bangladesh's software industry giving less priority on using test automation and deployment tools.
        
    \item \textbf{How are the security and performance ensured in a product of a company?}
        
    By this question, we try to understand how software companies secure and maintain their code by covering the dimensions like security, performance and scalability. The answer shows that standards are followed for the ensurance of security, tools are used for performance, and scalabiliy is ensured by using cloud services.
        
    \end{enumerate}
    
\item \textbf{Study  2. Comparison among Countries}: In this study we answer above questions involving comparison of software development practices between Bangladesh and other countries.

    \begin{enumerate}[label={\textbf{RQ\arabic{*}.}}, leftmargin=30pt]
    \item \textbf{What software methodology are used in your project?}
    
    In this comparison, we attempt to figure out in which development approaches Bangladesh differs from other countries. We have perceived that in practicing some notable approaches, for example most spent time, matured countries are far away. This comparison may help responsible persons to improve their current methodology to follow the standards.
    
    \item \textbf{Which implementation technologies and tools are adopted by software development professionals?}
    
    The use of technology and tools usually alters region to region. So we are interested to point out the similitude as well as different technologies and tools used in Bangladesh's software industry by comparing with industries of other countries.
    
    \item \textbf{What type of testing and deployment practices are used?}
    
    By this question, we compare testing practices of Bangladesh with other countries which shows that in test automation tools adoption, Bangladesh is way behind than few countries. This analysis may also help to increase the usage rate of automation tools.
    
    \item \textbf{How are the security and performance ensured in a product of a company?}
    
    We also aim to get the comparison in the ensurance of the security and performance in a software product by this research question. We have found both similarities and disimilarities in ensuring security, performance and scalability among countries.
    
    \end{enumerate}
    
\end{itemize}

Our findings show that Bangladeshi software companies usually spend most time on the implementation stage rather than system design and requirement analysis and also the use of automatic testing and tools for automatic deployment is quite low though now-a-days it is an integral part of a good and standard software product. In addition, we compare development processes of Bangladesh with other countries and analyze them by several dimensions. This findings can help both managers and researchers to invent new development environment to increase both benefit and growth rate of software industry. On top of that our study identifies the popularity of using cloud services which may advise cloud companies to spread their market more than ever in Bangladesh. Furthermore, our study points out the technologies and tools which are currently being used and popular in Bangladeshi software firms to assist both students and local universities for improving themselves to fill the gap of current needs of the IT sector.

\noindent\textbf{Paper Organization.} The rest of the paper is organized as follows. Section~\ref{related_works} presents the related work to our study. Section~\ref{study_setup} describes the background of our study and the data collection procedure. Section~\ref{study_results} reports the research questions about software development practices. Section~\ref{discussions} reports challenges and opportunities as well as analysis by several dimensions of software development practices. Section~\ref{implications} discusses the implications of our findings. Section~\ref{validity} discusses the threats to validity. Section~\ref{conclusion} concludes the paper.
