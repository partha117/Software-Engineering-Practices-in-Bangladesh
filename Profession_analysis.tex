\subsection{Analysis by Professions}
\label{analyze_by_professions}
% summarize key findings by professions. see Section 7.2 of \url{https://ieeexplore.ieee.org/abstract/document/8658125}
% In Tables \ref{table:analysis by profession part1}, \ref{table:analysis by
% profession part2}, and \ref{table:analysis by profession part3}, we summarize
% the interesting results of the closed questions by the reported roles of our survey participants. 
% For all questions, we reported the top values for each role. For
% example, for the question `Which of the following do you use for requirements
% gathering?' in Table \ref{table:analysis by profession part1}, we reported Plain
% Text for the developer's role. 26.2\% of software engineers use plain text for
% requirement gathering. Of all the other processes used by developers, the most
% commonly used requirement gathering process is plain text.

% % \begin{enumerate}[label=\arabic*)]
% %   \item Multiple Choice Questions. If the question has two options (Male, Female), we only show the result of the option(s) with the majority agreement.
% %   \item For all other questions, we reported the top two values for each role. For example, for the question ‘Which of the following do you use for requirements gathering?’ in Table \ref{table:analysis by profession part1}, we reported PT (Plain Text) and SB (Story Board) for the software engineer's role. 26.2\% of software engineers use PT, and 22.99\% use SB. These two requirements gathering processes are the top two among all the other processes used by software engineers.
% % \end{enumerate}

% % \newcolumntype{b}{X}
\newcolumntype{v}{>{\hsize=.04\hsize}X}
\newcolumntype{k}{>{\hsize=.96\hsize}X}
\begin{table}[!htbp]
    \centering
    \caption{Highlights of Findings from Survey Closed Questions by Profession}
    \begin{tabularx}{\textwidth}{v|k}
        \hline
        \textbf{No}     & \textbf{Question}  \\ \hline
        4         & For how many years have you coded professionally?\newline 1) Architect: 1) Architect: more than 10 (100\%) 2) Business analyst: more than 10 (100\%) 3) Data Engineer: more than 10 (100\%) 4) Developer: less than 2 (41.41\%) 5) Manager: more than 10 (58.33\%) 6) R\&D: less than 2 (100\%) 7) SQA: 2 to 5 (50\%) 8) Team Lead: 5 to 10 (100\%) 9) Trainer: 2 to 5 (100\%) 10) UXD: 5 to 10 (100\%)   \\ \hline
        
        
        6         & Which of the following software development methodologies do you follow?\newline 
        1) Architect: Agile (100\%) 2) Business analyst: Agile (50\%) 3) Data Engineer: Scrum (100\%) 4) Developer: Agile (60.82\%) 5) Manager: Agile (90\%) 6) R\&D: Agile (50\%) 7) SQA: Agile (87.5\%) 8) Team Lead: Agile (100\%) 9) Trainer: Agile (100\%)
        \\ \hline
        
        7 & Which of the followings do you use for requirements gathering?\newline
        1) Business analyst: GUI prototype (22.22\%) 2) Data Engineer: Plain Text (33.33\%) 3) Developer: Plain Text (26.2\%) 4) Manager: Grooming seasons (20.48\%) 5) R\&D: Grooming seasons (33.33\%) 6) SQA: Use Case (27.78\%) 7) Team Lead: GUI prototype (33.33\%) 8) Trainer: GUI prototype (25\%) 9) UXD: GUI prototype (50\%) \\ \hline
        
        8 & On which software development activities, do you spend most of the time?\newline
        1) Architect: Implementation (25\%) 2) Business analyst: Implementation (33.33\%) 3) Data Engineer: Implementation (33.33\%) 4) Developer: Implementation (30.33\%) 5) Manager: Requirement Analysis (22.37\%) 6) R\&D: Implementation (50\%) 7) SQA: Testing (25\%) 8) Team Lead: Documentation (25\%) 9) Trainer: Implementation (25\%) 10) UXD: Implementation (33.33\%) \\ \hline
        
    \end{tabularx}
    \label{table:analysis by profession part1}
\end{table}
% % \newcolumntype{b}{X}
\newcolumntype{v}{>{\hsize=.03\hsize}X}
\newcolumntype{k}{>{\hsize=.97\hsize}X}
\begin{table}[!ht]
    \centering
    \caption{Highlights of Findings from Survey Closed Questions by Profession}
    \begin{tabularx}{\textwidth}{v|k}
        \hline
        \textbf{No}     & \textbf{Question}  \\ \hline
        9 & Which of the following technologies do you have experience working in?\newline
        1) Architect: Mobile (50\%) 2) Business analyst: Desktop (33.33\%) 3) Data Engineer: Others (100\%) 4) Developer: Web (49.39\%) 5) Manager: Web (36.92\%) 6) R\&D: Web (66.67\%) 7) SQA: Web (40.91\%) 8) Team Lead: Desktop (50\%) 9) Trainer: Desktop (33.33\%) 10) UXD: Web (100\%)\\ \hline
        10 & What is the primary operating system you are developing on?\newline
        1) Architect: Linux (66.67\%) 2) Business analyst: Windows (66.67\%) 3) Data Engineer: MacOS (100\%) 4) Developer: Linux (42.86\%) 5) Manager: Linux (53.49\%) 6) R\&D: Linux (50\%) 7) SQA: Windows (56.25\%) 8) Team Lead: Windows (100\%) 9) Trainer: Others (100\%) 10) UXD: Linux (50\%) \\ \hline
        
        11 & Which programming languages are you using?\newline
        1) Architect: Java (100\%) 2) Business analyst: JavaScript (40\%) 3) Data Engineer: Go (25\%) 4) Developer: JavaScript (31.75\%) 5) Manager: JavaScript (27.06\%) 6) R\&D: Java (66.67\%) 7) SQA: Java (24\%) 8) Team Lead: C\# (25\%) 9) Trainer: C/C++ (25\%) 10) UXD: JavaScript (100\%) \\ \hline
        12 & Which frameworks are you using? \newline
        1) Architect: Spring (100\%) 2) Business analyst: ASP. NET (16.67\%) 3) Developer: Spring (26.61\%) 4) Manager: Spring (23.73\%) 5) R\&D: Spring (66.67\%) 6) SQA: ASP. NET (25\%) 7) Team Lead: ASP. NET (50\%) 8) Trainer: Django (100\%) \\ \hline
    \end{tabularx} 
    \label{table:analysis by profession part2}
\end{table}
% \newcolumntype{b}{X}
\newcolumntype{v}{>{\hsize=.03\hsize}X}
\newcolumntype{m}{>{\hsize=.2\hsize}X}
\newcolumntype{y}{>{\hsize=.33\hsize}X}
\begin{table}[]
    \centering
    \caption{Highlights of Findings from Survey Closed Questions by Profession}
    \begin{tabularx}{\textwidth}{v|b}
        \hline
           8 & On which software development activities, do you spend most of the time?\newline Program Design = PD, Requirement Analysis = RA, Impl. = Implementation
        {
        \begin{tabularx}{0.92\textwidth}{myy}
        \\
         & top1 (\%) & top2 (\%) \\
        Architect & Impl. (25.0)  & PD (25.0)  \\
        Business An. & Impl. (33.33)  & RA (33.33)  \\
        Data Eng. & Impl. (33.33)  & Maintenance (33.33)  \\
        Developer & Impl. (30.33)  & RA (18.85)  \\
        Manager & RA (22.37)  & Impl. (21.05)  \\
        R\&D & Impl. (50.0)  & PD (50.0)  \\
        SQA & Testing (25.0)  & RA (17.86)  \\
        Team Lead & Documentation (25.0)  & Impl. (25.0)  \\
        Trainer & Impl. (25.0)  & PD (25.0)  \\
        UXD & Impl. (33.33)  & Maintenance (33.33)  \\
        \end{tabularx}
        }\\ \hline
    
        9 & Which of the following technologies do you have experience working in?\newline
        {
        \begin{tabularx}{0.92\textwidth}{myy}
        \\
         & top1 (\%) & top2 (\%) \\
        Architect & Mobile (50.0)  & Web (50.0)  \\
        Business An. & Desktop (33.33)  & Mobile (33.33)  \\
        Data Eng. &  Data Engineering (50.0)  & Back End (50.0)  \\
        Developer & Web (49.39)  & Mobile (26.83)  \\
        Manager & Web (35.29)  & Mobile (27.94)  \\
        R\&D & Web (66.67)  & Mobile (33.33)  \\
        SQA & Web (40.91)  & Mobile (31.82)  \\
        Team Lead & Desktop (50.0)  & Web (50.0)  \\
        Trainer & Desktop (33.33)  & Mobile (33.33)  \\
        UXD & Web (100.0)  &  \\

        \end{tabularx}
        }\\ \hline
        
    \end{tabularx} 
    \label{table:analysis by profession part3}
\end{table}
\bf{Agile is the most practiced requirement gathering method. The popularity of the Agile method is consistent across the different reported roles in our surveys.} The second popular method is Scrum.
%\boxtext{Agile is the most practiced requirement gathering method. The popularity of the Agile method is consistent across the different reported roles in our surveys.}

In Q8, participants were asked to identify the SDLC activity, where most of the time is spent. Generally, it is expected that participants in the senior role (e.g., manager, team lead) will spend time in requirement analysis, documentation where participants in the junior role (e.g., developers, r\&d engineer) will spend time in implementation and testing. However, Q8's responses do not match our expectations. \bf{Across all roles, implementation is the most time spent activity in SDLC. However, testing is not considered one of the most time-consuming activities. Rigorous testing practices may not be prevalent in the Bangladesh SE industry.}
In most of the roles, Java is the most used languages (Java is the second most used language in cases where JavaScript is the most used language). Even in the data engineer role, Java is the most used language, though Python is mostly used in data processing. Like Java, Spring (one framework of Java) is the widely used framework regardless of role.
\bf{Java and JavaScript are the most used languages regardless of role.}
We observed from Q15 and Q13 that developers practice the highest level of automated testing, and developers mostly practice unit tests. One of the reasons for the high level of automated testing among developers may be that it is easier to achieve automated testing in unit testing due to different frameworks/libraries. Automated tests offering libraries/frameworks for other types of testing can encourage higher automated testing levels in other roles.
\bf{Developers mostly practice the highest level of automated testing. Automated testing frameworks for other types of testings may increase the practice of automated testing.}