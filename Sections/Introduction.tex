\section{Introduction}
\label{introduction}

%The software engineering (SE) industry in Bangladesh is considered relatively small compared to the size of its population (160 million-plus) and the size of the national economy. %Yet, the software industry in this country has started rapidly emerging and become part of Bangladesh's rapid development in recent years. 
%The majority of software companies were established after 2000 and have become a crucial resource for the economy of Bangladesh \cite{Basis2017}, and also the growth rate is expected to continue sharply. This propitious growth is bolstered by export trends and a large demand for IT automation in the domestic market. To continue this achievement by fulfilling the growing expectations of software consumers, the SE industry needs to follow standard practices in producing good quality software products. %Thus for this, we have to first point out how the software development process is exercised in such a rapidly emerging industry. 

%The practice of software development varies from region to region. A certain methodology may become more efficient in some regions but is not considered efficient in other regions. Although most of the factors behind software development are the same all the time, there are certain points that vary from region to region. For software development professionals, deciding on processes, practices, and techniques is critical. Sometimes, these decisions might be based on marketing and literature bias that supports the new or industry-supported practice.

%Many studies concerning the practices of software development have been conducted. Those studies convey similarities, but also differences in diverse regions, countries, and communities. Previous empirical studies have gone into how software development practices were conducted in North America, Europe, Turkey, New Zealand, etc. There are some prior studies on the Bangladesh software engineering (SE) industry, but the studies' focus and dimension were different. To the best of our knowledge, this is the first study to identify software engineering practices in emerging countries by taking the software industry in Bangladesh as a case study on how the software industry in an emerging environment works, the challenges they face, and the practices they adopt.
Bangladesh is a country of an emerging economy with a 160 million population. IT sector is considered a priority sector in Bangladesh over the last decade. The software development industry dominates this sector by providing the majority of local needs as well as exporting abroad. According to the Bangladesh Association of Software and Information Services (BASIS), 1100+ software companies operate in Bangladesh, where around 40\% have a global business. The foreign revenue earned by the industry is over 800 Million USD~\cite{BASIS2018}. Computer Science and Engineering (CSE) is the most demanding subject choice among the best students in the country. The strength of Bangladesh in this area has been manifested by continuous success in ACM International Collegiate Programming Contest (ICPC) for the last two decades. The software engineers from Bangladesh work in almost all the leading software companies of the world. To the best of our knowledge, whether the local software industry matches this achievement in terms of quality software development is yet to be formally investigated.

Bangladesh has to compete with neighboring countries of similar socio-economic backgrounds such as India, Pakistan, Sri Lanka, etc. who have established a footprint across the globe. The delivery of quality outcomes on time with a competitive budget is a major challenge. The practice of state-of-the-art SE methods and technologies is indispensable to achieve success. The purpose of this research is to systematically study and characterize the SE practices in Bangladesh, i.e., the tools and technology used for design, development, testing, and deployment. Similar studies were carried out in other countries such as Canada, Turkey, Netherlands, and New Zealand to have a mature understanding of the gap between SE research and actual practice in the industry~\cite{Garousi2013, Garousi2015, Vonken2012, Wang2018}. Hence, the current study also presents a comparative picture of SE practices in Bangladesh and other countries. 

For a software development company, the choice of technology evolves over time and varies widely according to application type. This study is likely to enable the software companies to make an informed choice as the industry practice is presented here. Another major benefit of such a study is having insight into different dimensions of industrial practice that help design better curriculum for current students and reveal the need for continuous education components for existing practitioners. 

This study aims to consider detailed activities beyond high-level practices gathered through surveying individual professionals. The survey was conducted in two phases, a limited primary survey with selected respondents and a final survey. The goal of the primary survey was to find any discrepancy or ambiguity in the survey questions. Also, the primary survey had an interview session with each respondent. From the findings of the interviews, the question of the survey was later adjusted. The survey is designed to reveal the methodology of software development practices, the adoption of technologies and tools by the professionals, and the use of testing and deployment practices in the company where the individual concerned works. For this purpose, we studied from two perspectives: (1) Understanding development practices in Bangladesh: We aim to perceive what methodologies, technologies, and testing practices are followed in software companies. (2) Comparison among the countries: We aim to figure out both similarities and dissimilarities in the development practices of Bangladesh with other countries, which would eventually reveal the areas where this industry needs improvement. 
\anindya{The two stages of the survey (manual interview and online), involvement of management - these may be discussed briefly.}\partha{Added}
More specifically, we answer the following research questions around the two studies:

\begin{itemize}[leftmargin=10pt]

\item \textbf{Study 1. Understanding development practices in Bangladesh}: We explore four research questions as follows:
    \begin{enumerate}[label={\textbf{RQ\arabic{*}.}}, leftmargin=30pt]
    
    \item \textbf{What software methodologies are used in your project?}
    
    This research question investigates the development approaches by reporting the information collected about development methodologies, requirements analysis processes, etc. The investigation of this question helps us to analyze the dominant practice of the industry. This particular investigation has manifested that the SE companies follow the agile model most to provide their software services, and the implementation stage usually acquires most time of the product deadline.
        
    \item \textbf{Which implementation technologies and tools are adopted by software development professionals?}
        
    There exist a plethora of technologies around the globe but using these vary from country to country. In this research question, we try to find out the current trending technologies like technology platforms, programming languages, frameworks, etc., in the software industries of Bangladesh that may help one who wants to pursue his career here. We have found that web-based software services are prevalent in the market, and JavaScript, being a popular language for web development, is mostly used among software practitioners.

\anindya{One line result summery is there for last 2 RQs. For consistency, something should be mentioned for the first two as well.}\khalid{added}        
        
    \item \textbf{What type of testing and deployment practices are used?}
        
    In this research question, we want to explore the present situation of testing and deployment practices adopted by the software firms in Bangladesh by investigating testing and deployment tools, test automation level, version control system, etc. This investigation enables us to find out if there is a need for improvement in the area of software quality assurance. We have observed that using test automation and deployment tools is not widespread in the industry.
        
    \item \textbf{How are the security and performance ensured in a product of a company?}
        
    In this research question, we try to understand how software companies secure and maintain their code and what practices are followed to ensure performance and scalability. The answer shows that standards are followed for security assurance, tools are used for performance, and scalability is ensured by using cloud services.
        
    \end{enumerate}
    
\item \textbf{Study 2. Comparison among the countries}: We conduct a comparative study to analyze the questions discussed above to identify differences in software development practices between Bangladesh and other countries.

    \begin{enumerate}[label={\textbf{RQ\arabic{*}.}}, leftmargin=30pt, start=5]
    \item \textbf{What software methodology are used in your project?}
    
    In this comparison, we attempt to figure out if Bangladesh differs significantly from other countries regarding development practices. This attempt would help software professionals to identify the methodologies which are not being followed, according to the practices that technologically matured countries exercise. Our attempt shows that Bangladeshi SE companies generally spend the most time on the implementation stage of the development, where technologically advanced countries prefer spending the most time on system design and planning.
    \anindya{The above two comments do not make much sense to me. Please rephrase.}\khalid{updated}
    
    \item \textbf{Which implementation technologies and tools are adopted by software development professionals?}
    
    The use of technology and tools usually varies from country to country due to the availability of experienced developers and the available budget. Hence, we are interested in pointing out if the companies in Bangladesh deviate from world standards for particular application domains.
    
    \item \textbf{What types of testing and deployment practices are used?}
    
     In this question, we compare the testing practices of Bangladesh with other countries, which shows in test automation tool adoption. Bangladesh is way behind technologically advanced countries. This analysis is expected to motivate practitioners to use automated tools for testing.
    
    \item \textbf{How are the security and performance ensured in the products of a company?}
    
     It is very challenging to ensure the security of software and to maintain high performance. Hence, this is very important to understand if the software companies in Bangladesh lag in this perspective compared to other countries.
    
    \end{enumerate}
    
\end{itemize}

Our findings show that the software companies in Bangladesh usually spend most of the time on the implementation stage rather than system design and requirement analysis. Also, the use of automated testing and tools for automatic deployment is quite low though nowadays, it is an integral part of a better and standard software development practice across the globe. Besides, we compare the development processes of Bangladesh with other countries and analyze them from several perspectives. These findings can help both managers and researchers to improve development practices in Bangladesh. On top of that, our study identifies the popularity of using cloud services, which may advise cloud companies to spread their market more than ever in Bangladesh or encourage private cloud development locally. Furthermore, our study presents the technologies and tools currently being used in Bangladesh's software firms that may help current students prepare themselves better for the industry.

\noindent\textbf{Paper Organization.} The remainder of the paper is organized as follows. Section~\ref{related_works} presents the related work to our study. Section~\ref{study_setup} describes the background of our study and the data collection procedure. Section~\ref{study_results} reports research questions about software development practices. Section~\ref{discussions} reports challenges and opportunities as well as analysis by several dimensions of software development practices. Section~\ref{implications} discusses the implications of our findings. Section~\ref{validity} discusses the threats to validity. Section~\ref{conclusion} concludes the paper.
