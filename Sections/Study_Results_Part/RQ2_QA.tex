\subsubsection{What type of testing and deployment practices are used?}

To get the picture of quality assurance (QA) in countries such as Malaysia, Turkey, France, etc., different studies have been carried out. In Table \ref{table:testing_comparison}, we compare the major outcomes of testing and deployment related responses from our study with the similar results of the previous studies. The comparative picture is discussed below.

\begin{itemize}
    % \item Requirements Clarity
    \item Software Testing Practices (Q14)
    \item Level of Automated Testing (Q15)
\end{itemize}

\begin{table}[]
\caption{Comparison of testing practices and test automation between our findings and prior findings}

\begin{tabular}{llll}

\hline
\multicolumn{1}{c}{\textbf{Practices}} & \multicolumn{1}{c}{\textbf{Our Study}} & \multicolumn{1}{c}{\textbf{Prior Study}} & \multicolumn{1}{c}{\textbf{Comparison}} \\ 
\hline 


\multicolumn{1}{l|}{\multirow{2}{*}{\parbox{0.1\textwidth}{What type of testing and deployment practices are used?}}
} 
& 
\multicolumn{1}{l|}{\multirow{2}{*}{\parbox{0.25\textwidth}{
\vspace{-50pt} (1) Around half of our respondents responded about carrying out unit testing and functional testing. Also, the utilization rate of acceptance and UI testing has an appreciable percentage in Bangladesh. (2) Automated testing exercise is not usual in Bangladesh as per the majority of our respondents.
}}
} 
& 
\multicolumn{1}{l|}{\comparisoncell{0.25}{\vspace{50pt}Unit testing is observed to be the most exercised in Malaysia (68.29\%) \citep{Baharom2006}, Canada (79.27\%) \citep{Garousi2013}, New Zealand (73\%) \citep{Wang2018} in a great percentage.
}}                                  

&

\multirow{2}{*}{\parbox{0.23\textwidth}{
\vspace{-70pt} We have perceived that unit testing is moderately practiced in Bangladesh, though its usage is comparatively spacious in other countries. In the adoption of test automation, the software industry of Bangladesh is way behind France though the usage rate might have similar to other European countries.
}} \\ \cline{3-3}

\multicolumn{1}{l|}{}                                       
& 
\multicolumn{1}{l|}{}                                       
& 

\multicolumn{1}{l|}{\comparisoncell{0.25}{
\vspace{50pt} Test automation is highly embraced by France software industry and comparatively less adopted in overall Europe~\citep{dutta1999}.
}} 
&                                         
\\ \hline

\end{tabular}
\label{table:testing_comparison}
\end{table}



\paragraph{Software Testing Practices}
From our study, as per \ref{fig:testing}, we see an interesting point that unit testing (53\%) and functional testing (49\%) are moderately used in Bangladesh, whereas from \cite{Garousi2013} and \cite{Wang2018} we can see that relatively a high percentage of their survey respondents in both Canada New Zealand rely on unit testing with 79.27\% and 73\% respectively. On the other hand, the adoption of acceptance testing and UI testing is quite similar to these countries. In Malaysia, based on \cite{Baharom2006}, Baharom et al. reported that, according to their survey, unit testing (68.29\%), integration testing (78.05\%), system testing (85.37\%), and acceptance testing (78.05\%) are used by most organizations in a high percentage, and about half of the organizations are carrying out alpha and beta testing.

\boxtext{Software developers around the world usually give unit testing the top priority, but the developers in Bangladesh have comparatively less participation.}
\rifat{Is the above observation correct? 53\% usage of unit testing reported.}\khalid{updated with percentage. The response rate of unit testing in our survey lags compares to others. Is it okay?}

\paragraph{Level of Automated Testing}
We have found that as per \ref{fig:autoTest}, around 25\% of our respondents are highly concerned that they have to use automated testing for their projects, \anindya{What does it imply by concerned?} \khalid{updated} while around 35\% of our respondents have expressed medium level concern and the remaining are hardly concerned about using automated testing. From the study of Dutta et al. \cite{dutta1999}, we have found that in automated testing practices, Bangladesh is quite similar to all of Europe but lags behind France. According to their study, the usage rate of automation testing tools in overall Europe is 26\%, where in France, it is as high as 61\%. But in Israel, this rate is the only 9\%.
