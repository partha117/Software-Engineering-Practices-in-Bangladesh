\subsubsection{Software development methodologies used (D1)}\label{sec:rq2-d1}
\begin{table}[h]
\caption{Comparison of development methodologies, requirements collection and activities between our findings and prior findings}
\begin{tabular}{llll}
\hline
\multicolumn{1}{c}{\textbf{Practices}}& \textbf{Our Study}& \textbf{Prior Study}&\multicolumn{1}{c}{\textbf{Comparison}}\\ \hline 


\multicolumn{1}{l|}{\multirow{3}{*}{\parbox{0.1\textwidth}{What software methodology are used in your project?
}}} 
& 
\multicolumn{1}{l|}{\multirow{3}{*}{\parbox{0.22\textwidth}{
(1) Agile and scrum are the two most used methods for development. (2) Appreciable respondents answered that plain text and storyboard for requirements gathering, which shows that tools for requirements collection yet are not used usually in Bangladesh. (3) The implementation stage gets more preference than system design, planning, and testing in Bangladesh.
}}} 
& 
\multicolumn{1}{l|}{\comparisoncell{0.31}{{\vspace{5pt}The usage of the scrum and waterfall is high in New Zealand \citep{Wang2018} and Turkey \citep{Garousi2015} respectively. Software developments in Malaysia are predominantly regulated through ad-hoc approach, and the agile methodologies \citep{Almomani2015}.
}}}
& 
\multirow{3}{*}{\parbox{0.23\textwidth}{
We have perceived that in the Bangladesh SE industry, there exists an ad-hoc based software development practice which might be similar to the practices in Malaysia nonetheless, the regions like Japan, Europe, etc., give more importance to system planning and design that makes those region-specific SE industries more stable.
}} \\ \cline{3-3}
\multicolumn{1}{l|}{}                                                                              & \multicolumn{1}{l|}{}                                                                                           
& \multicolumn{1}{l|}{\comparisoncell{0.31}{{\vspace{5pt}The textual description of specifying requirements is a firm favourite in Netherlands \citep{Vonken2012}.
}}}                                                         &                                                                                                                                                \\ \cline{3-3}
\multicolumn{1}{l|}{}                                                                              & \multicolumn{1}{l|}{}                                                                                                                                                                                                                                & \multicolumn{1}{l|}{\comparisoncell{0.31}{{\vspace{5pt} Specified design documentations are well-practiced rather than just implementation without well-defined planning and documentation in India, Japan, and Europe \citep{Cusumano2003}. However, most time is spent on implementation and coding in New Zealand \citep{Wang2018}.
}}} &                                                                        \\ \hline

\end{tabular}
\label{table:dev_methods_comparison}
\end{table}

We compare the observations from three questions in our survey: \begin{inparaenum}
\item Software development methodologies (Q6),
\item Requirements gathering (Q7), and
\item Most time-consuming software development activities (Q8).
\end{inparaenum} We summarize the comparisons in Table \ref{table:dev_methods_comparison} and discuss those below.

\nd\bf{$\bullet$ Software development methodologies (Q6).} Our study shows that
the most acceptable method in Bangladesh is the agile model (64\%) likewise
across the globe \citep{StackoverflowSurvey2018}. However, the usage of the
scrum (44\%) in New Zealand has better usage followed by agile (30\%)
\citep{Wang2018}, and in Turkey, the waterfall model is mostly used based on the
survey of Garousi et al. \citep{Garousi2015}. Again, in both Bangladesh and New
Zealand, extreme programming (XP) has a lower percentage of usage. Almomani et
al. \citep{Almomani2015} found that software developments in Malaysia are
predominantly regulated through ad-hoc approach (53\%) and the agile
methodologies (46\%) since usually software organizations are majorly concerned
with short-term delivery of software products.

\nd\bf{$\bullet$ Requirements gathering (Q7).} According to the
Figure~\ref{fig:requirements}, using plain text (44\%) and storyboard (41\%) are
the most widely used requirement gathering methods. This result is similar to
the survey of Vonken et al. \citep{Vonken2012}. From their study, we can find
that the textual description of specifying requirements is the most favorite
approach in the Netherlands.

\nd\bf{$\bullet$ Timeline of Development Activities (Q8).} According to the
study of Wang et al. \citep{Wang2018}, during system design and development,
most time is spent on implementation and coding, and relatively less time is
spent on maintenance in New Zealand similar to Bangladesh, as revealed in our
study. On the other hand, requirement analysis requires the second most time in
Bangladesh, according to 45\% respondents of our survey. In contrast, in
Malaysia, as per \citep{Baharom2006}, most organizations spend from 5\% to 20\%
of their efforts on planning and requirement analysis. However, if we compare
Bangladesh with technologically advanced regions like Japan, India, Europe,
etc., with the help of the study of Cusumano et al. \citep{Cusumano2003}, we
observe that there exist a substantial difference in the timeline of development
activities with Bangladesh. Their study has reported that architectural,
functional, and design specification documents are the most used and
well-regarded practice in those regions rather than just writing code with
minimal planning and documentation. But in Bangladesh, the implementation phase
gets the most priority over other development stages.
%, as presented in Figure
%\ref{fig:activities} of our study.


% To compare the industry of Bangladesh with that of other countries, we report the following sub-sections:
% 
% \begin{itemize}
% \item Software development methodologies (Q6).
% \item Requirements gathering (Q7).
% \item Most time-consuming software development activities (Q8).
% \end{itemize}
% 
% 
% 
% We have compared Bangladesh with countries of different geographical areas like Malaysia, Turkey, New Zealand, Europe, Japan, etc., on the criteria indicated above with the help of prior studies conducted in these countries. Thus we have got some similarities and dissimilarities in the practices of these countries. We compare the significant findings of our study with the relevant findings of prior studies in Table \ref{table:dev_methods_comparison}.
% 
% 
% \paragraph{Software development methodologies}
% Our study shows that the most acceptable method in Bangladesh is the agile model (64\%) likewise across the globe \citep{StackoverflowSurvey2018}. However, the usage of the scrum (44\%) in New Zealand has better usage followed by agile (30\%) \citep{Wang2018}, and in Turkey, the waterfall model is mostly used based on the survey of Garousi et al. \citep{Garousi2015}. Again, in both Bangladesh and New Zealand, extreme programming (XP) has a lower percentage of usage. Almomani et al. \citep{Almomani2015} found that software developments in Malaysia are predominantly regulated through ad-hoc approach (53\%) and the agile methodologies (46\%) since usually software organizations are majorly concerned with short-term delivery of software products.
% 
% 
% \paragraph{Requirements Gathering}
% According to the Figure~\ref{fig:requirements}, using plain text (44\%) and storyboard (41\%) are the most widely used requirement gathering methods. This result is similar to the survey of Vonken et al. \citep{Vonken2012}. From their study, we can find that the textual description of specifying requirements is the most favorite approach in the Netherlands.
% 
% 
% \paragraph{Development activities timeline}
% According to the study of Wang et al. \citep{Wang2018}, during system design and development, most time is spent on implementation and coding, and relatively less time is spent on maintenance in New Zealand similar to Bangladesh, as revealed in our study. On the other hand, requirement analysis requires the second most time in Bangladesh, according to 45\% respondents of our survey. In contrast, in Malaysia, as per \citep{Baharom2006}, most organizations spend from 5\% to 20\% of their efforts on planning and requirement analysis. However, if we compare Bangladesh with technologically advanced regions like Japan, India, Europe, etc., with the help of the study of Cusumano et al. \citep{Cusumano2003}, we observe that there exist a substantial difference in the timeline of development activities with Bangladesh. Their study has reported that architectural, functional, and design specification documents are the most used and well-regarded practice in those regions rather than just writing code with minimal planning and documentation. But in Bangladesh, the implementation phase gets the most priority over other development stages, as presented in Figure \ref{fig:activities} of our study.

\begin{tcolorbox}[flushleft upper,boxrule=1pt,arc=0pt,left=0pt,right=0pt,top=0pt,bottom=0pt,colback=white,after=\ignorespacesafterend\par\noindent]
\nd\it{\bf{RQ2-D1. Software development methodologies used.}}
The agile method is the most popular approach for software development across the globe except in some areas (e.g., New Zealand).
Bangladesh SE industry mainly uses plain text like other countries to collect requirements (e.g., Netherlands).
In comparison with technologically advanced regions, Bangladesh SE industry lags in giving value to system design and planning.
%\gias{summarize} 
\end{tcolorbox}