\begin{abstract}
\hfill
\newline
\noindent\textbf{Context:} The software engineering industry in Bangladesh offers a rapidly emerging environment and significantly contributes to the national economy. A thorough study to understand the software development practices is essential to figure out the challenges and opportunities to deliver high-quality software.
\newline
\textbf{Objective:} In this study, we aim to understand the methods and practices used in such a rapidly growing environment, what technologies and tools software development professionals utilize, how professionals assure the quality of products, and maintain software security and performance. We also try to compare Bangladesh with other countries on how practices vary from country to country based on this study's observed outcomes.
\newline
\textbf{Method:} In this study, we have conducted interview sessions and a survey. We have interviewed 8 individual participants from 4 leading software companies to reduce ambiguity from the survey questions. We have received 137 survey responses from the developers who are currently working in Bangladesh's software industry. 
\newline
\textbf{Results:} The major findings of our study on the companies of Bangladesh software industry are: i) Most companies follow the agile methodology and give implementation-related activities a high priority, ii) Most companies prefer web-based software products as a service, iii) Most companies use less automated testing and deployment tools in development, and iv) Several companies seriously consider the security, performance, and scalability of their software.
\newline
\textbf{Conclusion:} We hope that the aftermath of our study will help the software industry of Bangladesh to exercise standard software development processes for the overall improvement of quality. It will also enable both local software practitioners and foreign companies to prepare for the current circumstances of the industry.
\end{abstract}