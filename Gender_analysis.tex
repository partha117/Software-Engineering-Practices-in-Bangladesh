\subsection{Analysis by Gender}
\label{analysis by gender}
% summarize and compare the results by gender
% This paper may have something interesting regrading gender \urls{https://ieeexplore.ieee.org/document/9206232}
% \urls{https://ieeexplore.ieee.org/document/7965425}

In our survey, 90.1\% of participants were male, and 9.9\% participants were female, which is slightly better than the Stack Overflow (SO) survey \citep{StackoverflowSurvey2017,StackoverflowSurvey2018,StackoverflowSurvey2019,StackoverflowSurvey2020} (in Stack Overflow 8\% respondents marked them as female). It is often said that females are less represented in STEM. As the SE industry is directly related to STEM, the claim may be true in the SE industry. Though our survey does not represent the real scenario, the proportion of male and female participants supports the claim of under-representation. To get an overview of the Bangladesh SE industry by gender in Figure \ref{fig:gender and role}, we plotted the participants' roles grouped by gender. 
\begin{figure}[h]
\centering
 \includegraphics[scale=0.4]{Figures/Gender_and_Role}
 \caption{Gender based role}
 \label{fig:gender and role}
\end{figure}
In terms of roles, the proportion of developers among female participants is comparatively low than that of male participants. However, the scenario is the opposite of the manager and team lead role. Generally, the developer role is considered a junior role, and the manager/team lead is considered a senior role. Thus, it is clear from Figure \ref{fig:gender and role} that there is a difference between junior and senior roles among female participants. This indicates that in recent years females are less interested in joining the SE industry. Also, among ten different roles, female participants are holding only four types of roles. Our findings align with the survey result of Hussain et al.\citep{Hussain2020}. They found that female participants are only limited to developer, QA, and project manager roles, where male participants are holding varying types of roles in the Bangladesh SE industry. The result of the Stack Overflow survey\citep{StackoverflowSurvey2020} is slightly different. In the SO survey, female respondents are mostly data scientists, business analysts, QA, and developer roles. A common theme in all surveys is the role of female respondents in the QA.

\newcolumntype{b}{X}
\newcolumntype{v}{>{\hsize=.03\hsize}X}
\newcolumntype{m}{>{\hsize=.2\hsize}X}
\newcolumntype{y}{>{\hsize=.33\hsize}X}
\begin{table}[htbp]
    \centering
    \caption{Highlights of Findings from Survey Closed Questions by Gender}
    \begin{tabularx}{\textwidth}{v|b}
        \hline
        \textbf{No}     & \textbf{Question}  \\ \hline
        2 & What is your age?\newline
        1) 15 to 20: \textbf{\textit{Male, 100.0\% } } 2) 20 to 25: \textbf{\textit{Male, 87.8\% } } 3) 25 to 30: \textbf{\textit{Male, 86.84\% } } 4) 30 to 35: \textbf{\textit{Male, 96.0\% } } 5) 35 to 40: \textbf{\textit{Male, 100.0\% } } 6) 40 to 45: \textbf{\textit{Male, 100.0\% } } \\ \hline
        4 & For how many years have you coded professionally?\newline
        1) less than 2: \textbf{\textit{Male, 89.13\% } } 2) 2 to 5: \textbf{\textit{Male, 91.18\% } } 3) 5 to 10: \textbf{\textit{Male, 80.77\% } } 4) more than 10: \textbf{\textit{Male, 100.0\% } } \\ \hline
        5 & What is your current role?\newline
        1) Architect: \textbf{\textit{Male, 100.0\% } } 2) Business analyst: \textbf{\textit{Male, 100.0\% } } 3) Data Engineer: \textbf{\textit{Male, 100.0\% } } 4) Developer: \textbf{\textit{Male, 92.93\% } } 5) Manager: \textbf{\textit{Male, 95.83\% } } 6) R\&D: \textbf{\textit{Male, 100.0\% } } 7) SQA: \textbf{\textit{Male, 60.0\% } } 8) Team Lead: \textbf{\textit{Female, 100.0\% } } 9) Trainer: \textbf{\textit{Male, 100.0\% } } 10) UXD: \textbf{\textit{Male, 100.0\% } } \\ \hline
        6 & Which of the following software development methodologies do you follow?\newline
        1) Agile: \textbf{\textit{Male, 86.36\% } } 2) Kanban: \textbf{\textit{Male, 100.0\% } } 3) Others: \textbf{\textit{Male, 100.0\% } } 4) Pair Programming: \textbf{\textit{Male, 96.43\% } } 5) Scrum: \textbf{\textit{Male, 93.75\% } } 6) Waterfall: \textbf{\textit{Male, 100.0\% } } 7) XP: \textbf{\textit{Female, 100.0\% } } \\ \hline
        9 & Which of the following technologies do you have experience working in?\newline
        1) Cloud: \textbf{\textit{Male, 100.0\% } } 2) Desktop: \textbf{\textit{Male, 78.57\% } } 3) Embedded /IOT: \textbf{\textit{Male, 100.0\% } } 4) Mobile: \textbf{\textit{Male, 90.62\% } } 5) Others: \textbf{\textit{Male, 100.0\% } } 6) Web: \textbf{\textit{Male, 90.0\% } }\\ \hline
    \end{tabularx}
    \label{table:analysis by gender}
\end{table}

The interesting findings of gender analysis are presented in Table \ref{table:analysis by gender}. James et al.'s \citep{James2017} survey on male and female software professionals found that men are more likely to be in senior positions than women. In Q4 of Table \ref{table:analysis by gender}, we observe a similar scenario. However, in our case, the observation is not statistically significant. They also found that male software practitioners tend to be older than female practitioners and female practitioners tend to leave jobs in mid-career. We observe the same phenomena in Q2 and Q4, and the finding may be true in Bangladesh.


James et al.\citep{James2017} found that female practitioners express less satisfaction with the spirit of teamwork inside the organization. They conclude that this characteristic would make male practitioners a key part of a good agile team. Our findings differs from James et al. \citep{James2017}. In Q6 of Table \ref{table:analysis by gender}, we observe that our survey's female participants prefer the Agile methodology over other software development methodologies. However, development methodology selections are often performed by senior management. Thus, personal preferences can have minimal effect on this choice

In James et al.'s survey, neither male nor female was not dominating any particular technology field. However, in Q9 of 
Table \ref{table:analysis by gender} we observe female participants working on desktop, web, and mobile. As a small SE industry, cloud/IoT is considered a less secured field as job opportunities in these fields are small. It seems female participants tend to work in a more secure technology (in terms of a job opportunity) field.