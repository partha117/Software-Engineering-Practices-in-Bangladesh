\subsubsection{What type of testing and deployment practices are used?}
\label{testing_practices}

Testing is an important process in improving the quality of the software product. The purpose of this process is to find errors, which might occur during specification, design and code generation. Here we report the following results next:
\begin{itemize}
\item Software Testing Practices (Q 14)
\item Level of Automated Testing (Q 15)
\item Tools Used in Testing and QA (Q 16)
\item Continuous Deployment tools (Q 17)
\item Version Control (Q 18)
\end{itemize}


\paragraph{Software Testing Practices}
According to Figure \ref{fig:testing}, several number of testing practices are used during software development. The results show most of the organizations have carried out unit testing (53\%), functional testing (49\%), user acceptance testing (39\%), GUI testing (31\%) etc. We observed that managers perform GUI testing and performance testing, which is unlikely due to his role/designation. However, the observation is not statistically significant (p=0.12). To identify the relation between testing practices and experience, we plotted them together in Figure \ref{fig:testing type and experience}. In Figure \ref{fig:testing type and experience}, we observed that junior developers mostly perform unit, integration, and functional testing, where senior developers mostly perform API testing. We conducted the Mann Whitney U test to assess the conjecture, and it is statistically significant (p$<$0.01).
\begin{figure}[h]
\centering
  \includegraphics[scale=0.6]{Figures/Testing_Type_and_Experience}
  \caption{Testing practices ans professional experience}
  \label{fig:testing type and experience}
\end{figure}

\begin{figure}[h]
\centering
  \includegraphics[scale=0.2]{Figures/Respondents_testing_practices}
  \caption{Testing Practices}
  \label{fig:testing}
\end{figure}


\paragraph{Level of Automated Testing}
Question 15 asked about the level of automated testing. The responses were gathered using the Likert scale. This denotes that different respondents have very different practices in this context, i.e., some heavily practice automated testing, while others favor manual testing.  Results are shown in Figure \ref{fig:autoTest} which indicates that about 70\% of our respondents (others than who voted for level 5) do not use automated testing in a regular basis. The level of automated testing might be related to the programming language/framework. The testing suite provided by framework/language might encourage developers to implement automated testing. The level of automated testing vs language and framework is plotted in Figure \ref{fig:language and autotest} and \ref{fig:framework and autotest} respectively. It seems from Figure \ref{fig:language and autotest} that the highest level of automated testing mostly practices in Java, Javascript, Objective-C, and Php language. We conducted the Mann Whitney U test to assess our conjecture, and our conjecture is statistically significant (p=0.01). From Figure \ref{fig:framework and autotest}, we found that the highest level of automated testing is mostly performed in Android, Express, NodeJS, Struts, and Oracle Java EE framework, and the observation is statistically significant (p=0.006). Also, the highest level of automated testing is mainly used by developers (mostly use unit testing), and managers practice the lowest level of automated testing. The reason why managers use the lowest level of automated testing may be related to the type of testing they perform. We observed that managers mainly conduct GUI and performance tests. We guessed that experience might be one of the factors that influence automated testing. Our conjecture was senior developers might tend to use more automated tests than junior developers. We plotted Experience and automated test levels in Figure \ref{fig:experience and autotest}. However, we found the opposite scenario junior developers tend to use more automated testing than senior developers. However, the observation is not statistically significant ( p=0.08). One of the reasons behind this observation may be senior developers perform certain testings (e.g., GUI testing), which are hard to automate.

\begin{figure}[h]
\centering
  \includegraphics[scale=0.45]{Figures/Auto_Test_and_Experience}
  \caption{Experience and automated testing level}
  \label{fig:experience and autotest}
\end{figure}
\begin{figure}[h]
\centering
  \includegraphics[scale=0.65]{Figures/Language_and_Test_Level}
  \caption{Programming language and automated testing level}
  \label{fig:language and autotest}
\end{figure}
\begin{figure}[h]
\centering
  \includegraphics[scale=0.65]{Figures/Framework_and_Test_Level}
  \caption{Framework and automated testing level}
  \label{fig:framework and autotest}
\end{figure}

\begin{figure}[h]
\centering
  \includegraphics[scale=0.15]{Figures/Respondents_autotest_level}
  \caption{Automated Testing Level}
  \label{fig:autoTest}
\end{figure}

\boxtext{There exists a tendency among most of the Bangladeshi developers not using automated testing regularly.}


\paragraph{Tools Used in Testing and QA}
Q 16 asked about the tools used in testing and quality assurance. According to \ref{fig:testingTools}, we see that most of the respondents have been used XUnit( eg, JUnit, NUnit) (30\%), selenium (27\%), Jenkins (20\%), others (9\%). These results show that there exists a large demand of testing tools in the software industry of Bangladesh though around 38\% of our respondents were not interested to reply this question.

\begin{figure}[h]
\centering
  \includegraphics[scale=0.18]{Figures/Respondents_testing_tools}
  \caption{Testing \& QA Tools}
  \label{fig:testingTools}
\end{figure}

\boxtext{There has a large demand of software testing tools in Bangladesh.}


\paragraph{Deployment Tools}
According to \ref{fig:deployTools}, wee see that most of the respondents deploy their implemented codes using AWS code-deploy (12\%) and JenKins (12\%). The other deployment tools are Bamboo (5\%), teamcity (4\%), octopus (2\%). Respondents voted none (4\%) as they didn’t use any deployment tools and 53\% of the respondents were not interested about this topic. These outcomes show that usage rate of deployment tools in Bangladesh for continuous integration and continuous deployment is still in lower side yet it may increase in near future. We guessed that the practice of using deployment tolls might be seen only in senior developers. However, the hypothesis is not statistically significant (p=0.37).

\begin{figure}[h]
\centering
  \includegraphics[scale=0.18]{Figures/Respondents_deployment_tools}
  \caption{Deployment Tools}
  \label{fig:deployTools}
\end{figure}


\paragraph{Version Control}
% \hfill\\
Respondents were allowed to select more than one option. As shown in figure \ref{fig:versionControl}, Git (78\%) and Bit-Bucket (29\%) are mostly-used version control in the software industry. Beside these Subversion (SVN) (5\%), others (4\%) are used.  The 2018 Stack overflow survey\cite{StackoverflowSurvey2018} reports that  the most popular version control system is Git (87.2\% developer uses Git) and the second most popular is SVN (16.1\% developer uses SVN). However, in our survey, we found slightly different result,the most popular version control system is Git and the second most popular is Bit-bucket. This might be related to the declining popularity of SVN over the years. From the Stack overflow survey over the range 2017-2019, it is clear that SVN is losing popularity to Git. Nowadays, SVN is mainly used for versioning legacy projects. As the SE industry of Bangladesh is quite new, we observe such discrepancy in the popularity of SVN.

\begin{figure}[h]
\centering
  \includegraphics[scale=0.16]{Figures/Respondents_version_control}
  \caption{Version Control}
  \label{fig:versionControl}
\end{figure}
