\section{Study Setup}
\label{study_setup}
% \section{Study Setup}
% overview
% \subsection{Participants}
% how the participants are selected, what is the response rate? 
% any incentives were provided to encourage participation?
% \subsection{Data Collection and Analysis}
% how was the data collected (Google form), how was the data analyzed (open coding for open-ended questions, statistical analysis for closed questions) see \url{https://ieeexplore.ieee.org/abstract/document/8658125}

Our study has four steps as follows:
\begin{itemize}[leftmargin=10pt]
  \item\bf{Step 1 - Interview (Section \ref{sec:interview}).}  We conduct a series of semi-structured interviews to understand the current development practices in Bangladesh and to understand how 
  such practices could differ from other countries. 
  \item\bf{Step 2 - Survey (Section \ref{survey_participants}).} We design a survey questions to get deep insights on the insights collected from the interviews.  
  \item\bf{Steps 3, 4 - Data Analysis to answer RQ1 and RQ2 (Section \ref{survey_data_collection}).} We analyze the
  survey responses to answer RQ1 (understand development practices in
  Bangladesh) and RQ2 (compare the practices against other
  countries).
  
\end{itemize} %We discuss the steps below.
\subsection{Interview}\label{sec:interview}
The goal of
the interview session was to prepare the survey questions. Eight individual participants from four leading software companies were
interviewed. First we designed an initial list of survey questions in Google form by consulting previous studies in other countries like 
Canada, Turkey, Netherlands, New Zealand, etc.~\citep{Garousi2013, Garousi2015, Vonken2012, Wang2018}. Each
participant was asked
to identify ambiguities in the question. From the feedback of the
interview session, we revised the questions. Each interview session lasted about half an hour. 
Interviewees were first asked to complete the survey.
After completing the survey, we asked what he/she understood from the questions
and what he/she meant by the answers. Throughout the interviews, we identified
discrepancies between the understanding of the interviewee and the goals of the
survey questions by comparing the interview findings against findings from related work.
%Finally, we adjusted the survey questions to reduce ambiguity in survey questions.
\subsection{Survey Participants}
\label{survey_participants}

\newcolumntype{v}{>{\hsize=.08\hsize}X}
\newcolumntype{b}{>{\hsize=.77\hsize}X}
\newcolumntype{m}{>{\hsize=.15\hsize}X}
% \newcolumntype{y}{>{\hsize=.33\hsize}X}
\begin{table}[htbp]
    \centering
    \caption{The survey questions (without demographic questions) along the first two dimensions (D). Subscript with a question number shows number of responses.}
    \begin{tabularx}{\textwidth}{v|m|b}
        \hline
        \bf{RQ1 D1} & \bf{Question type} & \bf{Software development methodologies used} \\ 
        \midrule
        
        \multicolumn{1}{v|}{$6_{127}$} & \multicolumn{1}{m|}{Closed} & Which of the following software development methodologies do you follow? \\ 
        \cmidrule{2-3} 
        
        \multicolumn{1}{v|}{$7_{120}$} & \multicolumn{1}{m|}{Closed} & Which of the followings do you use for requirements gathering? \\ 
        \cmidrule{2-3} 
        
        \multicolumn{1}{v|}{$8_{126}$} & \multicolumn{1}{m|}{Closed} & On which software development activities, do you spend most of the time?\\ 
        \cmidrule{2-3} 
        
        \multicolumn{1}{v|}{\bf{RQ1 D2}} & \multicolumn{1}{m|}{\bf{Question type}} & \bf{Software tools and techniques used} \\
        \midrule
        
        \multicolumn{1}{v|}{$9_{128}$} & \multicolumn{1}{m|}{Closed} & Which of the following technologies do you have experience working in?\\ 
        \cmidrule{2-3}
        
        \multicolumn{1}{v|}{$10_{127}$} & \multicolumn{1}{m|}{Closed} & What is the primary operating system you are developing on?  \\
        \cmidrule{2-3} 
        
        \multicolumn{1}{v|}{$11_{128}$} & \multicolumn{1}{m|}{Closed} & Which programming languages are you using?\\ 
        \cmidrule{2-3} 
        
        \multicolumn{1}{v|}{$12_{109}$} & \multicolumn{1}{m|}{Closed} & Which frameworks are you using?\\
        \cmidrule{2-3} 
        
        \multicolumn{1}{v|}{$13_{125}$} & \multicolumn{1}{m|}{Closed} & Which IDE are you using?\\  \hline
        
    \end{tabularx}
    \label{table:survey_questions_1}
\end{table}
The survey questions are shown in Table \ref{table:survey_questions_1}. There are 17 questions, 14 closed and three open-ended.
We targeted developers who are currently working in the
software industry of Bangladesh. 
%
\newcolumntype{v}{>{\hsize=.08\hsize}X}
\newcolumntype{b}{>{\hsize=.8\hsize}X}
\newcolumntype{m}{>{\hsize=.12\hsize}X}
% \newcolumntype{y}{>{\hsize=.33\hsize}X}
\begin{table}[tbp]
    \centering
    \caption{The survey questions (without demographic questions) along the last two dimensions (D). Subscript with a question number shows number of responses.}
    \begin{tabularx}{\textwidth}{v|m|b}
        \hline
        \textbf{RQ1 D3} & \textbf{Question type} & \textbf{Software  testing  and  devops  practices used} \\ 
        \midrule
        
        \multicolumn{1}{v|}{$14_{117}$} & \multicolumn{1}{m|}{Closed} & What types of software testing practices do you use? \textit{(six options with text box only for `Others' option)} \newline \textbf{\textit{ 1) Unit testing 53.68\%, 2) User acceptance testing 39.71\%, 3) GUI testing 32.35\%, 4) Functional testing 49.26\%, 5) Performance testing 29.41\%, 6) Others 16.91\% } } \\
        \cmidrule{2-3} 
        
        \multicolumn{1}{v|}{$15_{118}$} & \multicolumn{1}{m|}{Closed} & What is the level of automated testing in your projects? \textit{(5-point Likert scale)} \newline \textbf{\textit{ 1) 1 11.76\%, 2) 2 13.24\%, 3) 3 17.65\%, 4) 4 17.65\%, 5) 5 27.21\% } } \\
        \cmidrule{2-3} 
        
        \multicolumn{1}{v|}{$16_{83}$} & \multicolumn{1}{m|}{Closed} & Which tools do you use for testing and quality assurance? \textit{(four options with text box only for `Others' option)} \newline \textbf{\textit{ 1) JenKins 19.85\%, 2) XUnit ( eg, JUnit, NUnit ) 30.15\%, 3) Selenium 27.21\%, 4) Others 12.5\% } } \\
        \cmidrule{2-3} 
        
        \multicolumn{1}{v|}{$17_{60}$} & \multicolumn{1}{m|}{Closed} & Which tools do you use for continuous deployment? \textit{(five options with text box only for `Others' option)} \newline \textbf{\textit{ 1) Bamboo 5.88\%, 2) TeamCity 4.41\%, 3) AWS codeDeploy 13.24\%, 4) Octopus 2.94\%, 5) Others 30.15\% } } \\ 
        \cmidrule{2-3} 
        
        \multicolumn{1}{v|}{$18_{121}$} & \multicolumn{1}{m|}{Closed} & Which version control tool do you use? \textit{(four options with text box only for `Others' option)} \newline \textbf{\textit{ 1) Git 76.47\%, 2) BitBucket 29.41\%, 3) Mercurial 2.21\%, 4) Others 9.56\% } } \\
        \cmidrule{2-3} 
        
        \multicolumn{1}{v|}{\textbf{RQ1 D4}} & \multicolumn{1}{m|}{\textbf{Question type}} & \textbf{Security  and  performance  measures used} \\ 
        \midrule
        
        \multicolumn{1}{v|}{$21_{74}$} & \multicolumn{1}{m|}{Open ended} & How do you ensure scalability of your products?                               \\ \cmidrule{2-3}
        \multicolumn{1}{v|}{$22_{73}$} & \multicolumn{1}{m|}{Open ended} & How do you maintain performance of your products?                               \\ \cmidrule{2-3}
        \multicolumn{1}{v|}{$23_{74}$} & \multicolumn{1}{m|}{Open ended} & How do you ensure security of your products?                               \\ \hline
        
        % \textbf{RQ1.2}     & \textbf{Question type} & \textbf{Software development methodologies}  \\ \midrule
        
        % \multicolumn{1}{v|}{1} & \multicolumn{1}{m|}{Closed} & \lipsum[1-1]                              \\ \cmidrule{2-3} 
        
    \end{tabularx}
    \label{table:survey_questions_2}
\end{table}
We applied purposive sampling~\citep{Vogt2005} to
include respondents in a software development related role. Purposive sampling
is basically based on the assumption of the population. It is possible that some
elements will not have a chance of selection in this method. Moreover, the
probability of selection can't be accurately determined in this process. We
shared the survey link through the authors' personal connection and in the local
developers' groups on social media to achieve our sampling goal. We also
implemented the chain referral strategy~\citep{creswell2013} and asked others to
pass on the survey invite. Due to such snowball approach to recruit survey participants, it is not possible to calculate the
response rate of our survey. We have conducted the survey through Google Forms. The survey link was opened
before the invitations were sent, and the survey link was closed for two
consecutive weeks without any response. The survey link was open for feedback
for about two months. In total, we
received 137 responses from the survey. Each participant was first asked a series of demographic questions (e.g., roles, experience, gender) and 
then was presented the survey questions related to the development practices. 
\tbl\ref{tab:role} shows the distribution of the survey participants by their roles. We noticed that a significant number (69\%) of our respondents are developers. Since software developers/developers is a generic role, this can be noticed in other surveys on the SE industry, like the Stakeoverflow survey\citep{StackoverflowSurvey2020,StackoverflowSurvey2019}. Previously conducted studies on the Canadian\citep{Garousi2013} and Turkish\citep{Garousi2015} SE industry found that more than 80\% of respondents were developers. Other roles for respondents to our survey are managers(16.9\%), and other kinds of software engineers (8\%) (e.g., data engineer, R\&D engineer).
\tbl\ref{tab:experience} shows the distribution of the survey participants by their experience. More than 61\% of the participants worked in the
industry for at least 2 years. In terms of work experience, the demographics of our survey are similar to previous surveys. 77\% of our respondents have less than ten years of experience. In the Canadian and Turkish SE industry survey, the percentage of respondents with less than ten years of experience is 67.9\% and  79\%, respectively. About 38.5\% of the respondents in our survey came from three software companies, while the rest (61.5\%) were from 38 
different companies. About half of the companies (51.9\%) are involved in web development, application development, and ERP software development. 
However, the companies offer a variety of products such as IoT-based health monitoring, cloud services, telecom, ride-sharing platform, biometrics-based personal identity management, and security solutions. According to BASIS, about 1100 software farms are employing 300000 IT professionals. BASIS has listed\citep{BASISList} the top ten firms in terms of revenue tax. This list indirectly presents the top player of the Bangladesh SE industry. 40\% of our respondents are from these software firms. Thus it can be said that our survey respondents represent the SE industry of Bangladesh.
%\gias{Please fill up by
%talking about some of the products the companies
%are developing. Use Linkedin for that)}.\partha{Updated} 
\begin{table}
\centering
\caption{Role wise Distribution of Participants}
\begin{tabular}{|c|c|}
\hline
\textbf{Role of the Participants} & \textbf{Percentage}\\
\hline
Developer & 69.72\%\\ 
Manager & 16.9\%\\ 
SQA Engineer & 7.04\%\\
Business Analyst & 1.4\%\\
R\&D Engineer & 1.4\%\\
Data Engineer & 0.7\%\\
Software Architect & 0.7\%\\
Team Lead & 0.7\%\\
Trainer & 0.7\%\\
UX Designer & 0.7\%\\
\hline
\end{tabular}
\label{tab:role}
\end{table}
\begin{table}[t]
\centering
\caption{Experience wise Distribution of Participants}
\begin{tabular}{|c|c|}
\hline
\textbf{Role of the Participants} & \textbf{Percentage}\\
\hline
less than 2 years & 33.58\%\\ 
2 to 5 years & 24.82\%\\ 
5 to 10 years & 18.98\%\\
more than 10 years & 17.52\%\\
experience not disclosed & 5.11\%\\
\hline
\end{tabular}
\label{tab:experience}
\end{table}

\gias{Can we say something about the representativeness of our survey population w.r.t. entire Bangladesh SE industry?}
\partha{We have not found any resource about the demographics of the Bangladesh SE industry. A few pieces of information are added.}
\gias{Can we compare the distribution of our survey demographics against similar surveys of other countries like NewZealand, Turkey, etc.? The purpose is to 
convince the reviewer that our survey population is representative compared to similar studies?}\partha{Added}
% 
% \subsection{Interviews}
% \label{interviews}
% For the interviews, we have conducted an interview session of about half an hour
% for each interviewee. Interviewees were first asked to complete the survey.
% After completing the survey, we asked what he/she understood from the questions
% and what he/she meant by the answers. From the interviews, we identified
% discrepancies between the understanding of the interviewee and the goals of the
% survey questions. The survey questions were then adjusted to reduce ambiguity
% among them.

\subsection{Data Analysis to Answer RQ1 and RQ2}
\label{survey_data_collection}
The survey consisted of both closed and open-ended questions. 
We analyzed the closed questions using standard descriptive and statistical techniques. We analyzed the 
closed questions following principles of open coding. Open coding includes labelling of concepts/
categories in textual contents based on the properties and
dimensions of the development entities (e.g., tools, processes, etc.) about which the contents
are provided. A systematic qualitative data analysis process was followed to analyze the
open-ended questions. First, the two authors independently coded the first 30\%
responses of each question's to extract potential categories. Second, the
authors conducted discussion sessions to develop a unified common coding scheme
for each question using these categories. Third, the rest of the responses were
coded using this coding scheme using the Coding Analysis Toolkit
(CAT)\citep{Lu2008} software. To measure the level of agreement between two
coders, we used the online tool Recal2\citep{Recal2020} and CAT\citep{Lu2008}. The
Recal2 calculator reports the agreement using four measures: 1) Percent
agreement, 2) Cohen $\kappa$\citep{Cohen1960}, 3) Scott’s $\pi$\citep{scott1955},
and 4) Krippendorff’s $\alpha$\citep{krippendorff2004}. It is believed that
Krippendorff’s $\alpha$ is more sensitive to bias introduced by a coder and is
recommended\citep{Joyce2013} over Cohen's $\kappa$\citep{Cohen1960}. The level of
agreement is presented in Table \ref{table: agreement level}. The average
$\kappa$ value was 0.71. It is a common practice that Cohen's $\kappa$ value
between 0.61 and 0.80\citep{Landis1977} is considered a `substantial agreement.’
In the coding process, a large number of codes were generated from each of the
open-ended questions. To help with our analysis, we conducted discussion
sessions to identify the codes that express similar themes. After reaching a
consensus, we grouped those codes into a smaller number of high-level
categories. We have used the statsmodels\citep{seabold2010} and
scipy\citep{scipy2020} modules in Python for statistical analysis. 
%\gias{can we say something about the distribution of the 137 participants by 
%the total number of software companies and the types of the software companies?}\partha{Added in the previous section. Should we present it with a chart? There is no significant difference in the type of companies; also, we do not know about most companies' type/product.}
% In the survey
% question, we had an optional question regarding the name of the company. Most of
% the participants answered the questions. Using the authors' personal connections
% and LinkedIn data, we have collected company size information from the company
% name. Later, we categorized the companies using the criteria presented in Table
% \ref{table: size criteria}.
% \partha{For open-ended questions, we have followed the methodology of the
% \href{https://link.springer.com/article/10.1007/s10664-019-09708-7}{blockchain
% paper}. The methodology of the opiner paper (`card') seems a bit
% different.}\partha{Other metrics of measuring agreement level is added}
\begin{table}[h]
\caption{The agreement level between the coders in the open
coding.}
\label{table: agreement level}
\centering
\begin{tabular}{cccc}
\hline
                & \textbf{Q21} & \textbf{Q22} & \textbf{Q23} \\ \hline
Percent         & 76.6         & 72.3         & 73.7         \\ \hline
Cohen   $\kappa$        & 0.731        & 0.688        & 0.71         \\ \hline
Scott  $\pi$         & 0.731        & 0.688        & 0.71         \\ \hline
Krippen  $\alpha$       & 0.732        & 0.689        & 0.711        \\ \hline
\textbf{Quotes} & 134          & 131          & 130          \\ \hline
\end{tabular}
\end{table}
%\begin{table}[t]
\centering
\caption{Size criteria}
\begin{tabular}{c|c}
\hline
\textbf{Category}   & \textbf{Size}      \\ \hline
Very Small & 1 - 20    \\ \hline
Small      & 21 - 50   \\ \hline
Medium     & 51 - 150  \\ \hline
Big        & 151 - 500 \\ \hline
Large      & 501 -       \\ \hline
\end{tabular}
\label{table: size criteria}
\end{table}
