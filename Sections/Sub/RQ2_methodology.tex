\subsubsection{Development methods and practices}
\label{dev_methods}

To find out the overall comparison of this section, we report the following sub-sections:

\begin{itemize}
\item Software development methodologies (Q 6).
\item Requirements gathering (Q 7).
\item Most time consuming software development activities (Q 8).
\end{itemize}

\paragraph{Software development methodologies}
From the study we see that the most acceptable model that was regularly and always used is the agile model (64\%) in Bangladesh but the usage of the scrum (44\%) in New Zealand has greater usage followed by agile (30\%) \cite{Wang2018} and in Turkey, waterfall is mostly used based on the earlier 2015 survey \cite{Garousi2015}. Again, in both Bangladesh and New Zealand, extreme programming (XP) has a lower percentage of usage.

\paragraph{Requirements Gathering}
According to \ref{fig:requirements}, using plain text (44\%) and story board (41\%) are the most widely used requirements gathering. This result is similar with the survey of Vonken et al. \cite{Vonken2012}. From their study we can find that the textual description of specifying requirements is a firm favourite in Netherlands.

\paragraph{Development activities timeline}
According to study \cite{Wang2018}, most time was spent on implementation and coding and also relatively less time was spent on maintenance in both Bangladesh and New Zealand. But requirement analysis, the activity, requires the second most time to spend in Bangladesh according to 45\% respondents where in New Zealand, it is testing (36\%) practices.
