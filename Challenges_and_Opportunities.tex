\subsection{Challenges and Opportunities}
\label{dicussion challanges}
% open ended question - scalability, security and performance
% now summarize the key finding - see Section 7.1 of \url{https://ieeexplore.ieee.org/abstract/document/8658125}
3.64\% of our respondents responded that they do not take any measures to mitigate the security risk. The common reason for not taking any measures is (1) the product is an early stage (2) the respondents' role does not require any product security measures. The reason for not taking any security measures is different from North America\cite{Assal2019}. The reasons are (1) there are no formal test plans, (2) lack of knowledge regarding testing tools. Though respondents do not think about product security initially, it is recommended\cite{Chandra2009,Azham2011} to plan security tests and product security from the design phase.

Modern frameworks provide the basic security of the solution. Moreover, some framework provides enhanced, focused, customized security through a plug-in or add-on, e.g., spring-security, spring-cloud security. Framework-based security is a growing practice in the software industry\cite{Alssir2012}. The practice in the Bangladesh SE industry matches the global practice. According to our survey, it is the second most popular measure to mitigate security risk. Survey respondents have reported using OWASP, HDIV, and spring security. Srinivasan et al.\cite{Srinivasan2017} have conducted a comparison among the popular web frameworks based on security. Based on five criteria, they ranked the frameworks, and all of the mentioned frameworks of our respondents are in the top 10 list. It seems that secure software engineering practice is prevalent in the Bangladesh SE industry.


On a survey of 237 software professionals, Elahi et al.\cite{Elahi2011} found that 51\% of respondents maintain at least one security standard, and 19\% of respondents maintain ISO 17799 security standard. On the contrary, about 10.22\% of respondents of the Bangladesh software industry maintain any security standard. It is clear that security standards are not that much popular in this SE industry.

According to Smith et al.\cite{Smith2003}, efficient architecture and continuous monitoring tools are two of the twenty-four best practices to ensure software performance. The respondents report both practices. However, Smith et al. presented twenty-one other best practices, and we have not found other practices in our survey. It is clear the SE industry of Bangladesh only practices a few best measures for ensuring software performance.

% 27.1\% of respondents in our survey reported that they use the load balancer to ensure software performance. A load balancer is mostly used to reduce queue time/ response time in web base application\cite{Mesbahi2016}. Thus we can infer that most of our survey respondents mainly engaged in developing the web-based application.


Bondi et al.\cite{Bondi2000} have listed four scalability types to ensure software capability to scale; however, we observe only one scalability type (load scalability) in our responses. To obtain software scalability use of cloud services is one of the most popular strategies\cite{Gao2011}. Another popular strategy is the use of microservice. Microservice and cloud services together allow the user to scale up and down any system dynamically. Cáceres et al.\cite{Cceres2010} reported that cloud services and microservice-based architecture are generally used together to ensure scalability. In the Bangladesh SE industry, this practice may be prevalent. The use of cloud service and efficient use of architecture are the second and third most popular topic among our respondents.