\subsubsection{How are the security and performance ensured in the products of a company?}

Several studies have been performed to identify security and performance practices in countries like Canada, New Zealand, and Turkey. We have found some differences as well as similarities in the practices of those countries with others. In Table \ref{table:open_ended_comparison}, we compare the major findings of our study against the findings of previous studies.

% Please add the following required packages to your document preamble:
% \usepackage{multirow}
\begin{table}[!ht]
\caption{Comparison of security, performance, scalability between our findings and prior findings}
\begin{tabular}{llll}

\hline
\multicolumn{1}{c}{\textbf{Practices}} & \multicolumn{1}{c}{\textbf{Our Study}} & \multicolumn{1}{c}{\textbf{Prior Study}} & \multicolumn{1}{c}{\textbf{Comparison}} \\ 
\hline

\multicolumn{1}{l|}{\multirow{3}{*}{\parbox{0.1\textwidth}{
How are the security and performance is ensured in a product of a company?
}}}

&

\multicolumn{1}{l|}{\multirow{3}{*}{\parbox{0.22\textwidth}{
(1) The practice of security architecture and security testing as a security measure are not prevalent in the Bangladesh SE industry. (2) Performance testing and peer review are the two least practiced performance measure in the Bangladesh SE industry. (3) Cloud service and architecture are two of the top three scalability measures.
}}} 

&

\multicolumn{1}{l|}{\comparisoncell{0.30}{
\vspace{13pt} Security testing is found to be the least practiced in the software industry in Turkey\citep{Garousi2015}, Malaysia\citep{Farvin2016}, India\citep{Bahl2011}, and New Zealand\citep{Sung2006}
}}                                                          

& 

\multirow{3}{*}{\parbox{0.22\textwidth}{
We found that the practice of `no security' measures is less in Bangladesh compared to other countries. Though the practice of security testing Bangladesh has similarities with other countries, it lags behind the practices of matured industries. However, like the matured and grown software industries, cloud service is a popular measure in Bangladesh.
}} 
\\ \cline{3-3}

\multicolumn{1}{l|}{}                                       
& 
\multicolumn{1}{l|}{}                                       
&

\multicolumn{1}{l|}{\comparisoncell{0.30}{
\vspace{13pt} Performance testing is a common practice in some \citep{Garousi2013,Garousi2015,Phillips2003} SE industry. However, it is hardly practiced in the Pakistan\citep{Jahan2019} SE industry. Also, peer review is a common practice among Turkish\citep{Garousi2015} developers.
}}                                                          

& 
\\ \cline{3-3}
\multicolumn{1}{l|}{}                                       
& 
\multicolumn{1}{l|}{}  
& 

\multicolumn{1}{l|}{\comparisoncell{0.30}{
\vspace{13pt} To ensure scalability, there are many practices prevalent in other countries' SE industry, such as micro-service architecture\citep{Laihonen2018} and containerization technologies\citep{Hussain2017}.
}} 
& 
\\ \hline


\end{tabular}
\label{table:open_ended_comparison}
\end{table}

\label{security_comparison}
A very few percentages (9.56\%) of our survey respondents reported not to use any security measures in their product. This practice is also prevalent in the Indian and Malaysian software industries. Bahl et al.\citep{Bahl2011} reported that due to misalignment with organization design, goal, and strategy in some Indian software firms, security measures are not practiced. In a study with Malaysian developers, Farvin et al.\citep{Farvin2016} found that 31\% of respondents think it is not required to add security in the requirement analysis of a product. Basharat et al.\citep{Basharat2013} reported a sense of false security in the small software industry and standard security practices are hardly followed. It is likely to be applicable to the industry in Bangladesh as well. From the response of a survey on the Turkish software industry, Garousi et al.\citep{Garousi2015} ranked different design activities in terms of frequency. Security architecture was ranked second out of five (five is for always used activities and one is for never used activities). The ranking represents that security architecture is not a frequent activity in the Turkish software industry. However, in our survey, we see very few respondents reported to practice security design principles while designing system architecture. Our survey found that 5.56\% of respondents rely on security architecture/security design principles (application side measures) to ensure security in their product. The software industries of Bangladesh, Turkey, and New Zealand have a resemblance in the practice of security testing. Garousi et al.\citep{Garousi2015} reported that security testing is least widely used among all kinds of testing (e.g., unit testing, integration testing). Sung et al.\citep{Sung2006} found that in the New Zealand software industry, security testing and recovery testing practices are low compared to functional testing. The scenario is the same for Bangladesh; we found that 16.67\% of respondents reported security testing to ensure security.


% \subsubsection{Performance measures}
% \label{performance_comparison}
21.82\% respondents of our survey use performance testing to ensure the performance of their product. \anindya{Should we say respondents or the companies they work? Think about it as we mention this at different places.}\partha{The survey question was `How do you ensure security of your products?' participants may answer this question from both personal and company perspective. It seems to me that the goal was to find out the participants' security practices. What do you suggest?} However, it is the second least practiced measure among all the measures. Garousi et al.\citep{Garousi2015} found that developers mark the lack of performance testing as the main challenge in software maintenance in the Turkish software industry. However, the scenario is different for the Canadian software industry. Participants of the survey of Garousi et al.\citep{Garousi2013} reported that 40\% of them conduct performance testing, and 30\% of their total testing effort is spent on performance testing. The New Zealand software industry follows the practice similar to Canada as reported by Phillips et al.\citep{Phillips2003}. %reported that performance testing is a common testing practice in New Zealand.
The practice in Bangladesh matches that of Pakistan. 
%In Pakistan software industry, performance testing is hardly practiced.
In the survey of Shah Jahan et al.\citep{Jahan2019}, only 5\% of participants reported conducting performance testing. It seems that performance testing is less popular in growing software industries such as Bangladesh and Pakistan.

Peer review is the least practiced measure in the Bangladesh software industry. However, in the Turkish software industry, peer review is ranked as the most frequent activity\citep{Garousi2015} (ranked five on a five-point Likert scale), though the practice is only limited to code review. Architecture/design review is hardly practiced in turkey (ranked one on a five-point Likert scale). We found that peer review is limited to only code review in the Bangladeshi software industry, and the only 7.27\% of our participants reported practicing peer review.

% \subsubsection{Scalability measures}
% \label{scalability_comparison}
There is no study focusing on scalability practices in specific software industries, so it isn't easy to compare scalability practices. However, in a study on Finnish DevOps, Laihonen\citep{Laihonen2018} found that the Finnish software industry prefers cloud services as it helps them automate quality assurance. He also reported that DevOps is inclined towards micro-service architecture when selecting a product rather than monolithic architecture. Hussain et al.\citep{Hussain2017} conducted a study to identify trends in the DevOps practices in New Zealand. For this study, besides interviewing the DevOps, they examined the job advertisements for a DevOps role. They found that containerization technologies (e,g., Docker, Kubernetes) have a high demand in the New Zealand software industry. 94\% of job advertisement requires expertise in one or multiple containerization technologies. This indicates the popularity of docker technology in the New Zealand software industry. However, in the Bangladeshi software industry, the scenario is different. Cloud services are the second most popular (28.85\%) measure to ensure scalability where the use of containerization technologies are not that much popular (3.85\%)

\begin{tcolorbox}[flushleft upper,boxrule=1pt,arc=0pt,left=0pt,right=0pt,top=0pt,bottom=0pt,colback=white,after=\ignorespacesafterend\par\noindent]
\nd\it{\bf{RQ2-D4. Security and performance measures used.}}
Compared to other emerging SE industries, Bangladesh has less practice of no security. However, we have noticed a lack of testing practice to ensure security and performance. Besides, the Bangladesh SE industry lags in using new technologies (e.g., Container, Cloud).
\gias{summarize RQ2-D4 here}
\partha{Added}
\end{tcolorbox}