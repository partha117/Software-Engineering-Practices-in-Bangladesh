\begin{abstract}
\hfill
\newline
\noindent\textbf{Context:}
%\anindya{Changed the context. Please check.}
A study on the software development practice in the industry of a country reveals the challenges and scopes of improvement for that industry and industries of similar types. There are several studies on the software industry of countries with an established reputation as a software manufacturer. However, to the best of our knowledge, there is no recent study that explores the issues and achievements of an emerging software industry. In this research, we focus on the picture of such industries and selected Bangladesh as a representative of this category.
%The software engineering industry in Bangladesh offers a rapidly emerging environment and significantly contributes to the national economy. A thorough study to understand the software development practices is essential to figure out the challenges and opportunities to deliver high-quality software.
\newline
\textbf{Objective:} We aim to understand the methods and practices used in the dynamic environment of software development, what technologies and tools the professionals utilize, how practitioners assure the quality of products, and maintain software security and performance. We also try to compare Bangladesh with other countries on how practices vary from country to country based on this study's observed outcomes.
\newline
\textbf{Method:} In this study, we have conducted interview sessions and a survey. We have interviewed 8 individual participants from 4 leading software companies to reduce ambiguity of the survey questions. Then we have conducted an online survey and received 137 responses from a diverse set of developers who are currently working in the software industry of Bangladesh. 
\newline
\textbf{Results:} The major findings of our study on the companies of Bangladesh software industry are: i) Most companies follow the agile methodology and give implementation-related activities a high priority, ii) Most companies are used to developing web-based software products, iii) Most companies use less automated testing and deployment tools, and iv) Several companies seriously consider the security, performance, and scalability of their software.
\newline
\textbf{Conclusion:} We hope that the outcome of our study will help the software industry of Bangladesh and similar countries formally understand the shortcomings compared to established industries and also help different stakeholders intervene accordingly. The students can prepare themselves to face the challenges of the industry in an informed way.
\anindya{Conclusion is also re-written.}
\end{abstract}