\section{Introduction}
\label{sec:introduction}
The term software engineering first appeared in the late 1960s and was introduced by Bauer to describe ways to develop, manage and maintain software so that the resulting products are reliable, correct, efficient and flexible. Some 15 years later, Zelkowitz et al. \cite{Zelkowitz1984} performed an in-depth survey of 30 companies in which they tried to establish the current state of practice in the software production industry. Their survey revealed that at that time practice was around 10 years behind on software engineering research. Almost 20 years later, in 2003 to be precise, Dr. Reifer observed that industry is a little behind academia, but industry has the capacity to close the gap very quickly\cite{Reifer2003}.


A software engineer applies engineering practices to software development, and typically handles the overall system design of the software application. Software engineering is divided into several specializations, which focus on a particular software development area. A few of the most common software engineering specialties include requirements engineering, software engineering processes, software design, tools and techniques. These specialities guide software engineering practice.


Software development varies from region to region. A certain methodology may become more efficient in some region which is not considered efficient in another regions. Though most of the factors behind software development are same all the time, there are certain points which vary from region to region. For software development professionals, deciding on processes, practices and techniques is critical. Sometimes, these decisions might be based on marketing and literature bias that supports new or industry-supported practice.

Many studies concerning the practices of the software development have been conducted. Those studies convey similarities, but also differences of diverse regions, countries and communities. Previous empirical studies have gone into how software development practices were conducted in the North America, Europe, Turkey and New Zealand but to the best of our knowledge, none were carried out in Bangladesh. Therefore, this study has been conducted to focus on the current practices of software development process and to provide perception of the software industries in Bangladesh. On top of that, the aspiration of this study is to look over the methodology of software development practices, the adoption of technologies and tools by professionals, and the use of testing and deployment practices. Specifically, we are interested in the developers' and managers' opinion about several questions regarding software development related practices. For example, i) \textit{What software methodology are used in your project?} ii) \textit{Which implementation technologies and tools are adopted by software development professionals?} iii) \textit{What type of testing and deployment practices are used?} iv) \textit{How are the security and performance is ensured in a product of a company?} and v) \textit{What are the personal expectations and desire reigning in the industry?}

Our study finds some similarities in development process of Bangladesh with other countries and also some interesting differences. For example, agile method is mostly used in Bangladesh but scrum has greater usage in New Zealand. On the contrary, textual description is used for requirements gathering in both Bangladesh in Netherlands. A lot number of developers of Bangladesh use frameworks for ensuring security and cloud services for scalability.

The rest of the paper is structured as follows. Section 2 presents previous related studies. Section 3 describes the demographics of the respondents and their projects. Section 4 presents the results obtained from this study and finally, Section 5 concludes the paper.
