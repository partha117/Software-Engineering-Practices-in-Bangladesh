\section{Implications of Findings}\label{implications}
The findings from our study can guide the following major stakeholders in software engineering (SE): 
\begin{inparaenum}[(1)]
\item SE Researchers to compare and contrast software development practices in emerging countries with respect to region-specific and global trends,   
\item SE Tool Creators to develop usable automated testing framework that can be affordable and accessible to the practitioners in emerging countries,     
\item SE career enthusiasts who would like to participate in the high-growth software industries in the emerging countries,    
\item SE security and performance practitioners to develop techniques to better enforce such crucial non-functional requirements in software products in emerging countries, and  
\item SE Industry leaders to offer customized region-specific software products and to promote diversity irrespective of regions.   
\end{inparaenum} We discuss the implications below.
% what are the implications of findings? say for another emerging country or software professionals or companies like Microsoft, Google?

\bf{\ul{SE Researchers.}}  We
have also found that testing has been given less priority in the industry though
it is one of the key elements for ensuring software quality. Furthermore,
automatic testing, which is crucial for the efficiency and productivity of the
test team, is rarely used. An automatic testing framework for other types of
testing (such as GUI testing) may increase automatic testing practice. The price
of quality automated testing tools may cause a lack of their use as the software
budget in developing countries is often limited. Researchers may create a
user-friendly framework to implement automatic testing for different types of
testing to address this affordability issue. This finding may encourage
researchers to develop open-source tools for the industry of developing
countries. Containerization technologies are a popular trend worldwide. However, the
practice of containerization technologies is too low. Automatic
deployment/release is another under-practiced area. Researchers can investigate
the challenges to incorporate these technologies in an emerging SE industry like
Bangladesh.

\bf{\ul{SE Testing Tool Creators.}} We have found that rigorous testing practice
is not prevalent in Bangladesh. The difference in testing effort between the
established software industry (e.g., Canada) and Bangladesh is too high. The
scenario is also true for other developing countries like Pakistan. From our
study, software practitioners may have an idea of their standing in software QA
in testing. Similar to all emerging industries, security testing is less
prioritized in Bangladesh. From our comparison, developers may better understand
security testing and security practices in other countries. Moreover, we have
shown the level of automatic testing in the Bangladesh SE industry. Automatic
testing is directly related to productivity. Practitioners can increase their
productivity by implementing automatic testing for their products. Similarly, we
have observed the under-use of automatic deployment/CI-CD tools. Practitioners
should focus more on this area to automate their product pipelines.
 

\bf{\ul{SE industry leaders.}} This study found that the Bangladesh software
industry lags in adopting some of the current industry trends. Bangladesh can be
a good marketplace for cloud companies like Microsoft, Google, and Amazon if
they provide an affordable package for the software companies operating here.
Local entrepreneurs may also think of building cost-effective local public
clouds along with providing some common DevOps services.

Moreover, we have found most of the participants in our study are engaged in web
technology. This may indicate that the web is the most popular form among users
of Bangladesh. This observation can help the industry owners to select
appropriate medium while targeting users of this region. %We have identified
common practices of the practitioners of the Bangladesh software industry.

In this study, we have identified that the
participation of female in the Bangladesh SE industry is comparatively less than
that of male engineers (see Section \ref{analysis by gender}). Hussain et al.\citep{Hussain2020} expressed a
concern that there may be bias in the hiring process of the industry. Further
research may be conducted to identify the inherent reason behind these facts.

\bf{\ul{SE Career Enthusiasts.}} We have found that certain languages (e.g.,
Java, JavaScript, etc.) and frameworks (e.g., Spring, Django, ASP.NET) have
extensive use in the software industry of Bangladesh. Universities can update
their curricula to meet industry demand. The students aspiring to join the
software industry must prepare themselves accordingly to be productive quickly.
We have also found less automated testing practices in the industry of
Bangladesh. This may be related to a lack of exposure to the testing framework.
Similar to Hussain et al.\citep{Hussain2020}, we also suggest including
testing-related courses into the curriculum of universities and introducing
relevant assignments to have hands-on experience on automated testing tools
right from the student level. The students in their academic projects should
also use container and other DevOps tools to bring some qualitative improvement
for the industry when they join there.

\bf{\ul{SE security and performance practitioners.}} 9.56\% of our respondents
responded that they do not take any measures to mitigate the security risk. The
common reason for not taking any measures is (1) the product is an early stage
(2) the respondents' role does not require any product security measures. The
reason for not taking any security measures is different from North
America\citep{Assal2019}. The reasons are (1) there are no formal test plans,
(2) lack of knowledge regarding testing tools. Though respondents do not think
about product security initially, it is recommended\citep{Chandra2009,Azham2011}
to plan security tests and product security from the design phase.

Modern frameworks provide the basic security of the solution. Moreover, some
framework provides enhanced, focused, customized security through a plug-in or
add-on, e.g., spring-security, spring-cloud security. Framework-based security
is a growing practice in the software industry\citep{Alssir2012}. The practice
in the Bangladesh SE industry matches the global practice. According to our
survey, it is the second most popular measure to mitigate security risk. Survey
respondents have reported using OWASP, HDIV, and spring security. Srinivasan et
al.\citep{Srinivasan2017} have conducted a comparison among the popular web
frameworks based on security. Based on five criteria, they ranked the
frameworks, and all of the mentioned frameworks of our respondents are in the
top 10 list. It seems that secure software engineering practice is prevalent in
the Bangladesh SE industry.


On a survey of 237 software professionals, Elahi et al.\citep{Elahi2011} found
that 51\% of respondents maintain at least one security standard, and 19\% of
respondents maintain ISO 17799 security standard. On the contrary, about 29.63\%
of respondents of the Bangladesh software industry maintain any security
standard. It is clear that security standards are not that much popular in this
SE industry.

According to Smith et al.\citep{Smith2003}, efficient architecture and
continuous monitoring tools are two of the twenty-four best practices to ensure
software performance. The respondents report both practices. However, Smith et
al. presented twenty-one other best practices, and we have not found other
practices in our survey. It is clear the SE industry of Bangladesh only
practices a few best measures for ensuring software performance.

% 27.1\% of respondents in our survey reported that they use the load balancer
% to ensure software performance. A load balancer is mostly used to reduce queue
% time/ response time in web base application\citep{Mesbahi2016}. Thus we can
% infer that most of our survey respondents mainly engaged in developing the
% web-based application.


Bondi et al.\citep{Bondi2000} have listed four scalability types to ensure
software capability to scale; however, we observe only one scalability type
(load scalability) in our responses. To obtain software scalability use of cloud
services is one of the most popular strategies\citep{Gao2011}. Another popular
strategy is the use of microservice. Microservice and cloud services together
allow the user to scale up and down any system dynamically. Cáceres et
al.\citep{Cceres2010} reported that cloud services and microservice-based
architecture are generally used together to ensure scalability. In the
Bangladesh SE industry, this practice may be prevalent. The use of cloud service
and efficient use of architecture are the second and third most popular topic
among our respondents.
